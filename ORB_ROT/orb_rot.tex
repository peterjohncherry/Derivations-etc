\documentclass[12pt]{article}
\usepackage[utf8x]{inputenc}
\usepackage[english]{babel}
\usepackage[T1]{fontenc}
\usepackage{color}
\usepackage{wrapfig}
\usepackage{amsmath}
\usepackage{amssymb}
\usepackage{textcomp}
\usepackage{array}
\usepackage{booktabs}
\usepackage{subfigure}
\usepackage[font=small,format=plain,labelfont=bf,up,textfont=it,up]{caption}
\usepackage{longtable}
\usepackage{calc}
\usepackage{setspace}
\usepackage{multirow}
\usepackage{hhline}
\usepackage{ifthen}
\usepackage{lscape}
\usepackage{relsize}
\usepackage{bbold}
\usepackage{mathtext}
\usepackage{pdfpages}
\usepackage{geometry}
 \geometry{
 a4paper,
 total={170mm,257mm},
 left=35mm,
 top=25mm,
 bottom=25mm,
 right=20mm,
 }
\usepackage{cite}
\linespread{1.3}

\begin{document}

Have multi-configurational wavefunction:

\begin{equation}
|M\rangle = \sum_{I} c_{I} |I\rangle
\end{equation}
\begin{equation}
= |M\rangle = c_{i_{1},i_{2}....i_{N}} a^{\dagger}_{i_{1}}a^{\dagger}_{i_{2}}...a^{\dagger}_{i_{N}}  |0\rangle
\end{equation}
\noindent  A second quatized version of a component of an orbital rotation operator is
\begin{equation}
\hat{V}_{ij}  = V_{ij}a^{\dagger}_{i}a_{j}
\end{equation}
A transformations,$\hat{U}$, of the Hamiltonian can be constructed from a rensor product of rotations in reals space, $\hat{S}$, and another in real space $\hat{R}$:
\begin{equation}
\hat{U} = \hat{R}\otimes\hat{S} 
\end{equation}
\noindent  $\hat{R}$ and $\hat{S}$ can be constructed from transformation of the spatial and spin components of the orbitals.
Aa basis formed from configuration state functions can be decomposed into a set subspaces, each of which are invariant under spin transformations $\hat{S}$, e.g.,,
\begin{equation}
[\hat{U}]
\begin{bmatrix}
[\Phi_{s_{-1}r_{1}}]\\
[\Phi_{s_{0}r_{1}}]\\
[\Phi_{s_{1}r_{1}}]\\
[\Phi_{s_{-1}r_{2}}]\\
[\Phi_{s_{0}r_{2}}]\\
[\Phi_{s_{1}r_{2}}]\\
...         .
\end{bmatrix}
=
[\hat{R}]
\begin{bmatrix}
[\hat{S}]
\begin{bmatrix}
[\Phi_{s_{-1}r_{1}}]\\
[\Phi_{s_{0}r_{1}}]\\
[\Phi_{s_{1}r_{1}}]\\
\end{bmatrix}
\\
[\hat{S}]
\begin{bmatrix}
[\Phi_{s_{-1}r_{2}}]\\
[\Phi_{s_{0}r_{2}}]\\
[\Phi_{s_{1}r_{2}}]\\
...         .
\end{bmatrix}
\end{bmatrix}
\end{equation}
\noindent i.e., the rotations $\hat{R}$ only rotate between different CSFs, whilst the rotations, $\hat{S}$,  rotate within multiplets.
Accordingbly, a roitation in $\hat{R}$ is enough to block diagonalize the Hamiltonian. 


\noindent The $\gamma_{ij}^{M,I}$ derivatives have matrix elements $\sum_{J}\langle I | a^{\dagger}_{i} a_{j} | J \rangle c_{M,I} $.  It would be 
nice to find a rotation $Q$ such that
\begin{equation}
\begin{bmatrix}
Q
\end{bmatrix}
\begin{bmatrix}
c_{M,1}\\
c_{M,1}\\
...\\
...\\
c_{M,N_{det}}\\
\end{bmatrix}
\end{equation} 
 

\end{document}
