\documentclass[12pt]{article}
\usepackage[utf8x]{inputenc}
\usepackage[english]{babel}
\usepackage[T1]{fontenc}
\usepackage{color}
\usepackage{wrapfig}
\usepackage{amsmath}
\usepackage{amssymb}
\usepackage{textcomp}
\usepackage{array}
\usepackage{booktabs}
\usepackage{subfigure}
\usepackage[font=small,format=plain,labelfont=bf,up,textfont=it,up]{caption}
\usepackage{longtable}
\usepackage{calc}
\usepackage{setspace}
\usepackage{multirow}
\usepackage{hhline}
\usepackage{ifthen}
\usepackage{lscape}
\usepackage{relsize}
\usepackage{bbold}
\usepackage{mathtext}
\usepackage{pdfpages}
\usepackage{geometry}
 \geometry{
 a4paper,
 total={170mm,257mm},
 left=35mm,
 top=25mm,
 bottom=25mm,
 right=20mm,
 }
\usepackage{cite}
\linespread{1.3}

\begin{document}


The motivation for asking is that one of the people in this group has been
doing loads of calculations with 4 component stuff, and is not really getting
much improvement over MRSCF DKH2 results.

Similarly, whilst I often found ReSpect-mDKS got better results than 2c DFT, I
didn't really find it was much of an improvement over MR methods with DKH2. In
your papers, and most other places, the comparison is usually 4c DFT vs 2c DFT,
not 4c DFT vs 2c MRSCF. That's not intended as a criticism; the former
comparison is obviously much more sensible. However, whenever I do see 4c DFT
vs 2c MRSCF , there's usually not much difference. This has me worried about
what the point of doing 4c MRSCF for the following reason (this is really over
simplified, but I think the logic is pretty reasonable):

In the following,
The error in the electron correlation for DFT is $C_{DFT}$ .\\
The error in the electron correlation for MRSCF is $C_{MR}$ .\\
The error in the relativity effects for 2C methods is $R_{2c}$ .\\
The error in the relativity effects for 4C methods is $R_{4c}$ .\\

Generally speaking, $C_{DFT} > C_{MR}$ and $R_{2c}>R_{4c}$. \\

So either
\begin{equation*}
C_{DFT} +R_{2c} <  R_{2c}+C_{MR} < R_{4c}+C_{DFT} < C_{MR} +R_{4c}
\end{equation*}
\begin{equation*}
C_{DFT} +R_{2c} < R_{4c}+C_{DFT} <  R_{2c}+C_{MR}  <C_{MR} +R_{4c}
\end{equation*}

Which suggests the quality of results should go like

\begin{equation*}
2c DFT  < 2c MRSCF  = 4c DFT < 4c MRSCF
\end{equation*}

But what I'm seeing.
2c DFT  < 2c MRSCF  = 4c DFT = 4c MRSCF

4c, non-collinear, unrestricted MOs should enable a better description of
multireference wavefunctions, and I'm worried that this might actually be the
main reason why ReSpect-mDKS does better than 2c DFT.

In other words, use of 4c shrinks E_corr by an amount

 I'm probably wrong, but this would explain why 4c MRSCF in Bagel isn't getting
better results than 2c MRSCF. Do you know of any papers which show clear
examples which show me I'm wrong? 
