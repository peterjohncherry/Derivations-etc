\documentclass[12pt]{article}
\usepackage[utf8x]{inputenc}
\usepackage[english]{babel}
\usepackage[T1]{fontenc}
\usepackage{color}
\usepackage{wrapfig}
\usepackage{amsmath}
\usepackage{amssymb}
\usepackage{textcomp}
\usepackage{array}
\usepackage{booktabs}
\usepackage{subfigure}
\usepackage[font=small,format=plain,labelfont=bf,up,textfont=it,up]{caption}
\usepackage{longtable}
\usepackage{calc}
\usepackage{setspace}
\usepackage{multirow}
\usepackage{hhline}
\usepackage{ifthen}
\usepackage{lscape}
\usepackage{relsize}
\usepackage{bbold}
\usepackage{mathtext}
\usepackage{pdfpages}
\usepackage{geometry}
 \geometry{
 a4paper,
 total={170mm,257mm},
 left=35mm,
 top=25mm,
 bottom=25mm,
 right=20mm,
 }
\usepackage{cite}
\linespread{1.3}

\begin{document}
\begin{equation*}
\sum_{L} \sum_{M}\sum_{J}\sum_{\substack{abcd\\ wxyz}} 
\langle J |\tilde{c}^{M\dagger}_{J} T^{\dagger}_{LM,abcd}(\hat{E}^{\dagger}_{ab}
\hat{E}^{\dagger}_{cd}-\delta_{bc}\hat{E}_{ad})(\hat{E}_{wx}\hat{E}_{yz}-\delta_{xy}\hat{E}_{wz})g(1,2)|K\rangle
\end{equation*}

\begin{equation}
=\sum_{L} \sum_{M}\sum_{J}\sum_{\substack{abcd\\ wxyz}} 
\langle J |\tilde{c}^{M\dagger}_{J}T^{*}_{LM,badc}\hat{E}_{ba}\hat{E}_{dc}\hat{E}_{wx}\hat{E}_{yz}g(1,2)|K\rangle
\end{equation}
\begin{equation}
-\sum_{L} \sum_{M}\sum_{J}\sum_{\substack{abcd\\ wxyz}} 
\delta_{xy}\langle J |\tilde{c}^{M\dagger}_{J}T^{*}_{LM,badc}\hat{E}_{ba}\hat{E}_{dc}\hat{E}_{wz}g(1,2)|K\rangle
\end{equation}
\begin{equation}
-\sum_{L} \sum_{M}\sum_{J}\sum_{\substack{abcd\\ wxyz}} 
\delta_{bc}\langle J |\tilde{c}^{M\dagger}_{J}T^{*}_{LM,badc}\hat{E}_{ad}\hat{E}_{wx}\hat{E}_{yz}g(1,2)|K\rangle
\end{equation}
\begin{equation}
+\sum_{L} \sum_{M}\sum_{J}\sum_{\substack{abcd\\ wxyz}} 
\delta_{bc}\delta_{xy}\langle J |\tilde{c}^{M\dagger}_{J}T^{*}_{LM,badc}\hat{E}_{ad}\hat{E}_{wz}g(1,2)|K\rangle
\end{equation}
\\
\noindent Define a map between a set of orbital indexes $\{a,b,c,d,w,x,y,z\}$ and a determinant index:
\begin{equation*}
\Pi^{MK} : \{a,b,c,d,w,x,y,z\} \rightarrow J
 \text{ \ \ \ \ \ \ \ \ \ \ } | J\rangle = \hat{E}_{ab}\hat{E}_{cd}\hat{E}_{wx}|K\rangle 
\end{equation*}
\begin{equation*}
\pi^{MK} (a,b,c,d,w,x,y,z) =\pi^{MK}_{abcdwxyz}= J 
\end{equation*}
\begin{equation*}
\Theta^{MK}_{abcdwxyz} = c_{\pi^{MK}_{abcdwxyz}} = c_{J} 
\end{equation*}
These expressions can be used to rewrite the above expressions without a sum over $J$:
\begin{equation*}
\text{(1) = } \sum_{L} \sum_{M}  \sum_{\substack{abcd\\ wxyz}}\Theta^{MK}_{abcdwxyz} \langle \pi^{MK}_{abcdwxyz} |T^{*}_{LM,badc}\hat{E}_{ba}\hat{E}_{dc}\hat{E}_{wx}\hat{E}_{yz}g(1,2)|K\rangle
\end{equation*}
Note that $|J\rangle$ and $|K\rangle$ are orthogonal
any to determinants with orbitals in the virtual space occupied\footnote{assuming preservation of the number of electrons},
or orbitals in the inactive space unoccupied. This enables further simplification of the above expressions via grouping together
different ranges, i.e., inactive, active, virtual, for the indexes, $a,b,c,d$, of the MS-CASPT2 amplitudes, $T_{LM,abcd}$:\\

\noindent If $a,b,c,d \notin \{active\} $, then $|K\rangle = \hat{E}_{xw}\hat{E}_{zy}|J\rangle$, hence 
\begin{equation*}
\sum_{L} \sum_{M} \sum_{abcd}^{\notin \{act\} } \sum_{wxyz}^{\in\{act\}} \Theta^{MK}_{abcdwxyz} \langle \pi^{MK}_{abcdwxyz} |T^{*}_{LM,badc}\hat{E}_{ba}\hat{E}_{dc}\hat{E}_{wx}\hat{E}_{yz}g(1,2)
\hat{E}_{xw}\hat{E}_{zy}|K\rangle
\end{equation*}
\begin{equation*}
=\sum_{L} \sum_{M} \sum_{abcd}^{\notin \{act\} } \sum_{wxyz}^{\in\{act\}} \Theta^{MK}_{abcdwxyz} \langle \pi^{MK}_{abcdwxyz}|T^{*}_{LM,badc}\hat{E}_{ba}\hat{E}_{dc}g(1,2)|\pi^{MK}_{abcdwxyz} \rangle
\end{equation*}
\begin{equation*}
=\sum_{L} \sum_{M}  \sum_{wxyz}^{\in\{act\}} \Theta^{MK}_{wxyz} \sum_{abcd}^{\notin \{act\} }T^{*}_{LM,badc}g(badc)
\end{equation*}
\begin{equation*}
=\sum_{L} \sum_{M}  \sum_{wxyz}^{\in\{act\}} \Theta^{MK}_{wxyz}A^{LM}_{wxyz}
\end{equation*}
\\
\noindent If $a,b \notin \{active\} $ and $c,d \in \{active\}$, then $|K\rangle = \hat{E}_{dc}\hat{E}_{xw}\hat{E}_{zy}|J\rangle$,  so 
\begin{equation*}
\sum_{L} \sum_{M} \sum_{ab}^{\notin \{act\} } \sum_{cdwxyz}^{\in\{act\}} 
\Theta^{MK}_{abcdwxyz} \langle \pi^{MK}_{abcdwxyz}|T^{*}_{LM,badc}\hat{E}_{ba}\hat{E}_{dc}\hat{E}_{wx}\hat{E}_{yz}g(1,2)\hat{E}_{cd}\hat{E}_{xw}\hat{E}_{zy} |\pi^{MK}_{abcdwxyz}  \rangle
\end{equation*}
\begin{equation*}
=\sum_{L} \sum_{M} \sum_{ab}^{\notin \{act\} } \sum_{cdwxyz}^{\in\{act\}} \sum_{u}^{\in\{d,x,z\}}\sum_{v}^{\in\{c,w,y\}}
\Theta^{MK}_{abcdwxyz} \langle \pi^{MK}_{abcdwxyz}|T^{*}_{LM,badc}\hat{E}_{ba}g(1,2)\hat{E}_{uv} |\pi^{MK}_{abcdwxyz}  \rangle
\end{equation*}
\begin{equation*}
=\sum_{L} \sum_{M}\sum_{cdwxyz}^{\in\{act\}}\Theta^{MK}_{cdwxyz}   \sum_{ab}^{\notin \{act\} } \sum_{u}^{\in\{d,x,z\}}\sum_{v}^{\in\{c,w,y\}}
 \langle \pi^{MK}_{abcdwxyz}|T^{*}_{LM,badc}\hat{E}_{ba}g(1,2)\hat{E}_{uv} |\pi^{MK}_{abcdwxyz}  \rangle
\end{equation*}
\begin{equation*}
=\sum_{L} \sum_{M}\sum_{cdwxyz}^{\in\{act\}}\Theta^{MK}_{cdwxyz}   \sum_{ab}^{\notin \{act\} } \sum_{u}^{\in\{d,x,z\}}\sum_{v}^{\in\{c,w,y\}}T^{*}_{LM,badc}g(bauv)
\end{equation*}
\begin{equation*}
=\sum_{L} \sum_{M}\sum_{cdwxyz}^{\in\{act\}}\Theta^{MK}_{cdwxyz}A^{LM}_{cdwxyz}
\end{equation*}
\noindent If $a,b,c,d \in \{active\} $  then $|K\rangle = E_{\gamma\delta\eta\zeta}|J\rangle$ and  
\begin{equation*}
\sum_{L} \sum_{M} \sum_{abcdwxyz}^{\in\{act\}} 
\Theta^{MK}_{abcdwxyz} \langle \pi^{MK}_{abcdwxyz}|T^{*}_{LM,badc}\hat{E}_{ba}\hat{E}_{dc}\hat{E}_{wx}\hat{E}_{yz}g(1,2)\hat{E}_{ab}\hat{E}_{cd}\hat{E}_{xw}\hat{E}_{zy} |\pi^{MK}_{abcdwxyz}  \rangle
\end{equation*}
\begin{equation*}
=\sum_{L} \sum_{M} \sum_{abcdwxyz}^{\in\{act\}} \Theta^{MK}_{abcdwxyz}\sum_{\substack{s,u\\ s\neq u}}^{\in\{b,d,x,z\}}\sum_{\substack{t,v\\ t\neq v}}^{\in\{a,c,w,y\}}
 \langle \pi^{MK}_{abcdwxyz}|T^{*}_{LM,badc}\hat{E}_{uv}g(1,2)\hat{E}_{tv} |\pi^{MK}_{abcdwxyz}  \rangle
\end{equation*}
\begin{equation*}
=\sum_{L} \sum_{M}\sum_{cdwxyz}^{\in\{act\}}\Theta^{MK}_{cdwxyz}   \sum_{ab}^{\notin \{act\} } \sum_{u}^{\in\{d,x,z\}}\sum_{v}^{\in\{c,w,y\}} T^{*}_{LM,badc}g(stuv)
\end{equation*}
\begin{equation*}
=\sum_{L} \sum_{M}\sum_{abcdwxyz}^{\in\{act\}}\Theta^{MK}_{abcdwxyz}A^{LM}_{abcdwxyz}
\end{equation*}

\end{document}
%Define a map between a set of orbital indexes $\{a,b,c,d...\}$ and a determinant index $J$:
%\begin{equation*}
%\Pi^{MK} : (a,b,c,d,w,x,y,z) \rightarrow J \text{\ \ \ \ such\ that \ \ \ } |J\rangle =   \hat{E}_{ba}\hat{E}_{dc}\hat{E}_{wx}\hat{E}_{yz}|K\rangle
%\end{equation*}
%\begin{equation*}
%\pi^{MK} (a,b,c,d,w,x,y,z) =\pi^{MK}_{abcdwxyz}= J 
%\end{equation*}
%\begin{equation*}
%\Theta^{MK}_{abcdwxyz} = c_{\pi^{MK} (a,b,c,d,w,x,y,z)} = c_{J} 
%\end{equation*}
%%\begin{equation*}
%w,x,y,z \in \{active\text{\ } indices\}
%\end{equation*}
%\begin{equation*}
%i,j,k,l \in \{inactive, \text{\ } active, \text{\ } virtual  \text{\ }   indices\}
%\end{equation*}
%Define a map between a set of orbital indexes $\{i_{1},i_{2},i_{3}i_{4}...\}$ and a determinant index:
%\begin{equation*}
%\Pi^{MK} : \{i_{1},i_{2},i_{3}i_{4}...i_{n}\} \rightarrow J \text{\ \ \ \ such\ that \ \ \ } |J\rangle =   \hat{E}_{ba}\hat{E}_{dc}|K\rangle
%\end{equation*}
%\begin{equation*}
%\pi^{MK} (a,b,c,d,w,x,y,z) =\pi^{MK}_{abcdwxyz}= J 
%\end{equation*}
%\begin{equation*}
%\Theta^{MK}_{abcdwxyz} = c_{\pi^{MK} (a,b,c,d,w,x,y,z)} = c_{J} 
%\end{equation*}
%\
%satisfying $|K\rangle = \hat{E}_{\kappa\lambda}\hat{E}_{\mu\nu}|J\rangle$ with $\{\kappa,\lambda,\mu,\nu\} \subset  \{a,b,c,d,w,x,y,z\}$
