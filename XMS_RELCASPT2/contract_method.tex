\documentclass[12pt]{article}
\usepackage[utf8x]{inputenc}
\usepackage[english]{babel}
\usepackage[T1]{fontenc}
\usepackage{color}
\usepackage{wrapfig}
\usepackage{amsmath}
\usepackage{amssymb}
\usepackage{textcomp}
\usepackage{array}
\usepackage{booktabs}
\usepackage{subfigure}
\usepackage[font=small,format=plain,labelfont=bf,up,textfont=it,up]{caption}
\usepackage{longtable}
\usepackage{calc}
\usepackage{setspace}
\usepackage{multirow}
\usepackage{hhline}
\usepackage{ifthen}
\usepackage{lscape}
\usepackage{relsize}
\usepackage{bbold}
\usepackage{mathtext}
\usepackage{pdfpages}
\usepackage{geometry}
 \geometry{
 a4paper,
 total={170mm,257mm},
 left=35mm,
 top=25mm,
 bottom=25mm,
 right=20mm,
 }
\usepackage{cite}
\linespread{1.3}

\begin{document}
Have unpleasant expression:
\begin{equation}
\sum_{\substack{ ijkl \\ mnop } }\sum_{J} \langle I | ijk^{\dagger}l^{\dagger}m^{\dagger}n^{\dagger}op|J\rangle c_{J}^{N}H_{ijkl}^{\dagger}G_{mnop}
\end{equation}

\noindent where $a_{i}^{\dagger} = i^{\dagger}$. Reorder to normal ordering to ensuer no non-active indexes in BraKet.

\begin{equation}
\sum_{\substack{ ijkl \\ mnop } }\sum_{J} \langle I | k^{\dagger}l^{\dagger}m^{\dagger}n^{\dagger}ijop|J\rangle c_{J}^{N}H_{ijkl}^{\dagger}G_{mnop}
\end{equation}

\noindent Reorder back to ordering where indexes corresponding to like operators are grouped:

\begin{equation}
\sum_{\substack{ ijkl \\ mnop } }\sum_{J}\langle I | ijk^{\dagger}l^{\dagger}m^{\dagger}n^{\dagger}op|J\rangle c_{J}^{N}H_{ijkl}^{\dagger}G_{mnop}
\end{equation}

\noindent    If no contractions between operators, perform the contractions between operator representation and rdms/rdms derivatives seperately:

\begin{equation}
\sum_{ ijkl  }H_{ijkl}^{\dagger}\sum_{K} \langle I | ijk^{\dagger}l^{\dagger}|K\rangle
                         \sum_{mnop}\sum_{J}G_{mnop}\langle K | m^{\dagger}n^{\dagger}op|J\rangle c_{J}^{N}
\end{equation}

\noindent If there are some contractions between different operators can use a similar approach

\begin{equation}
\sum_{ ijkl  }H_{ijkl}^{\dagger} \sum_{K}\langle I | ijk^{\dagger}|K\rangle
                         \sum_{nop}\sum_{J}G_{lnop}\langle K | n^{\dagger}op|J\rangle c_{J}^{N}
\end{equation}

\noindent Can evaluate this using the following sequence of operations:

\begin{itemize}
\item $ \sum_{J}   \langle K | n^{\dagger}op|J\rangle c_{J}^{N} = \sigma_{n^{\dagger}op}^{K}$
\item $ \sum_{nop} G_{lnop}\sigma_{n^{\dagger}op}^{K} = \tilde{x}_{l}^{K}$
\item $\sum_{l} H_{ijkl}^{\dagger} \tilde{x}_{l}^{K} = \tilde{w}_{ijk}^{K} $
\item $ \sum_{ijk}\sum_{K} \langle I | ijk^{\dagger}|K\rangle \tilde{w}^{K}_{ijk} = \tilde{c}^{I} $ 
\end{itemize}

\noindent This avoids the need to perform the contractions or perform index
shuffling on large combined operators, e.g., shuffling on  $A_{ijkmno} =
\sum_{l} H^{\dagger}_{ijkl}G_{lnop}$, and also eliminate the need to store
rdms/rdm derivatives with more than 4 indexes. However, it is
worth noting that the contracted index, $l$ in the above example, is often not
active, and hence $\tilde{x}_{l}^{K}$ is usually either $n_{core}n_{det}$ or
$n_{virt}n_{det}$, which is typically $ >> n_{act}n_{det}$. However, so long as
$n_{virt} < n_{act}^{4} $ the size of    $\tilde{x}_{l}^{K} < \sigma_{ijkl}^{K} $,
so $\tilde{x}_{l}^{K}$ is still unlikely to be the largest matrix we have to deal with.
\\

\noindent A disadvantage is that it requires repeated performance of the
operations of the sort $a^{\dagger}_{i}a_{j}a_{k}|J\rangle$.  And the fact that
the determinant space used in the RI ($\{|K\rangle\}$ in the above example) is not the space
of active determinants. 
\end{document}
