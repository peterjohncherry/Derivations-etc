\documentclass[12pt]{article}
\usepackage[utf8x]{inputenc}
\usepackage[english]{babel}
\usepackage[T1]{fontenc}
\usepackage{amsmath}
\usepackage{amssymb}
\usepackage{textcomp}
\usepackage{array}
\usepackage{relsize}
\usepackage{bbold}
\usepackage{mathtext}
\usepackage{pdfpages}
\usepackage{geometry}
 \geometry{
 a4paper,
 total={170mm,257mm},
 left=15mm,
 top=15mm,
 bottom=15mm,
 right=15mm,
 }

\usepackage{cite}
\linespread{1.3}

\begin{document}
Define space 
\begin{equation}
\mathrm{s} = \mathrm{p} + \mathrm{s} = \{ |p\rangle \} \cup \{|q\rangle \}
\end{equation}
Have projectors
\begin{equation}
\hat{P} =\sum_{p} |p\rangle \langle p | 
\text{ \ \ \ \ \ \ \ \ \ }
\hat{Q} =\sum_{q} |q\rangle \langle q | 
\end{equation}
So...
\begin{equation}
\hat{P}+\hat{Q} = \hat{I}
\text{\ \ \ \ \ \ \ \ \ }
\hat{P}-\hat{I} = \hat{Q}
\end{equation}
Then have zeroth order Hamiltonian, $\hat{H}_{0}$, and perturbation, $\hat{V}$. The spaces $\mathrm{p}$ and $\mathrm{q}$ are such that
\begin{equation}
\langle p | \hat{H}_{0} | q \rangle  = 
\langle q | \hat{H}_{0} | p \rangle  = 
\langle p | \hat{V} | p \rangle  =  0 \text{ \ \ \ } \forall \text{ \ }p\text{,\ }q  . 
\label{eqn:space_props}
\end{equation}
Write two equations for the zeroth order, $\hat{H}_{0}$, and full, $\hat{H} = \hat{H}_{0} + \hat{V}$, Hamiltonians:
\begin{equation}
\hat{H}_{0} | p \rangle  = \epsilon_{p} |p \rangle
\end{equation}
\begin{equation}
\hat{H}|\Psi_{p} \rangle =( \hat{H}_{0}+\hat{V} )| \Psi_{p} \rangle  = (\epsilon_{p}+E_{p})|\Psi_{p} \rangle
\end{equation}
Here, $|\Psi_{p}\rangle$, is being treated as though it were a perturbation of state $|p\rangle$. Initially, 
assume that there is only one state in the space $\mathrm{p}$. It is now more convenient to write this as
\begin{equation}
(\epsilon_{p}-\hat{H}_{0} )| \Psi_{p} \rangle  = (\hat{V}-E_{p})|\Psi_{p} \rangle
\end{equation}
The wavefunction $|\Psi_{p} \rangle$ may be expanded in $\mathrm{p}$ and $\mathrm{q}$;
\begin{equation}
| \Psi_{p} \rangle  = \sum_{p} C_{p}|p\rangle + \sum_{q} T_{q}|q\rangle.
\end{equation}
Hence,
\begin{equation}
\hat{P} | \Psi_{p} \rangle  = \sum_{p} C_{p}|p\rangle
\text{ \ \ \ \  and \ \ \ \ }
\hat{Q} | \Psi_{p} \rangle  = \sum_{q} T_{q}|q\rangle .
\end{equation}
This, combined with the identities associated with the projectors $\hat{P}$ and $\hat{Q}$, and the
attributes of the space, $\mathrm{s}$, stated in (\ref{eqn:space_props}) leads to the following statement
\begin{equation}
\hat{P}(\epsilon_{p} - \hat{H}_{0})(\hat{P}+\hat{Q})| \Psi_{p} \rangle  =  \hat{P}(\hat{V}-E_{p})(\hat{P}+\hat{Q})|\Psi_{p} \rangle .
\label{eqn:P_on_both sides}
\end{equation}
Taking each side seperately:
\begin{equation}
\hat{P}(\epsilon_{p} - \hat{H}_{0})(\hat{P}+\hat{Q})| \Psi_{p} \rangle  = 0  ,
\label{eqn:P_on_EH}
\end{equation}
\begin{equation}
\hat{P}(\hat{V}-E_{p})(\hat{P}+\hat{Q})|\Psi_{p} \rangle = \hat{P}\hat{V}\hat{Q}|\Psi_{p} \rangle + \hat{Q}E_{p}\hat{Q}|\Psi_{p}\rangle .
\label{eqn:P_on_VE}
\end{equation}
Substutiting (\ref{eqn:P_on_EH}) and (\ref{eqn:P_on_VE}) back into (\ref{eqn:P_on_both sides}) and rearranging we obtain
\begin{equation}
\hat{P}\hat{V}\hat{Q}|\Psi_{p} \rangle =  \hat{P}E_{p}\hat{P}|\Psi_{p}\rangle .
\label{eqn:pt_energy}
\end{equation}
Assuming the members of $\mathrm{p}$ are orthonormal and taking advantage of the consequent idempotency of $\hat{P}$ leads to
\begin{equation}
\hat{P}E_{p}\hat{P}|\Psi_{p}\rangle = E_{p}\hat{P}|\Psi_{p} \rangle ,
\end{equation}
hence
\begin{equation}
\sum_{p'} \langle p' | E_{p}\hat{P}|\Psi_{p} \rangle  =  \sum_{p'} \langle p' | \hat{V}\hat{Q}|\Psi_{p} \rangle.
\label{eqn:multistate_pt_energy}
\end{equation}
Whilst the primary intention of the MRPTool module is to tackle multistate, highly degenerate cases, it helpful\footnote{for me, at least} 
to first consider the singlestate case, i.e., where the dimenation of $\mathrm{p}$ is 1. 
If intermediate normalization, $\langle p | \Psi_{p} \rangle = 1$, is assumed then 
\begin{equation}
\hat{P}E_{p}\hat{P}|\Psi_{p}\rangle = E_{p},
\end{equation}
which leads to 
\begin{equation}
E_{p} = \langle p | \hat{V}\hat{Q}|\Psi_{p} \rangle.
\label{eqn:singlestate_pt_energy}
\end{equation}
If in (\ref{eqn:P_on_both sides}) $\hat{Q}$ instead of $\hat{P}$ was applied from the left the following expressions would result:
\begin{equation}
\hat{Q}(\hat{V}-E_{p})(\hat{P}+\hat{Q})|\Psi_{p} \rangle = \hat{Q}\hat{V}|\Psi_{p} \rangle - \hat{Q}E_{p}\hat{Q}|\Psi_{p}\rangle ,
\label{eqn:Q_on_VE}
\end{equation}
\begin{equation}
\hat{Q}(\epsilon - \hat{H}_{0})(\hat{P}+\hat{Q})| \Psi_{p} \rangle  = \hat{Q}\epsilon_{p}\hat{Q}|\Psi\rangle- \hat{Q}\hat{H}_{0}\hat{Q}|\Psi_{p}\rangle.
\end{equation}
However, noting that the RHS of (\ref{eqn:P_on_EH}) is $0$ it is possible to write
\begin{equation*}
\hat{Q}(\epsilon_{p} - \hat{H}_{0})(\hat{P}+\hat{Q})| \Psi_{p} \rangle =
(\hat{P} + \hat{Q})(\epsilon_{p} - \hat{H}_{0})(\hat{P}+\hat{Q})| \Psi_{p} \rangle =
(\epsilon_{p} - \hat{H}_{0})| \Psi_{p} \rangle ,
\end{equation*}
hence
\begin{equation}
(\epsilon - \hat{H}_{0})| \Psi_{p} \rangle  = \hat{Q}\epsilon_{p}\hat{Q}|\Psi\rangle- \hat{Q}\hat{H}_{0}\hat{Q}|\Psi_{p}\rangle.
\label{eqn:Q_on_EH} 
\end{equation}
Equating the RHS's of (\ref{eqn:Q_on_EH}) and (\ref{eqn:Q_on_VE})
 \begin{equation*}
(\epsilon_{p} - \hat{H}_{0})|\Psi_{p}\rangle
= \hat{Q}\hat{V}|\Psi_{p} \rangle - \hat{Q}E_{p}\hat{Q}|\Psi_{p}\rangle .
\end{equation*}
If there is only one state in $\mathrm{p}$, then it is possible to substitute in from  (\ref{eqn:singlestate_pt_energy}) to obtain
\begin{equation*}
( \epsilon_{p}- \hat{H}_{0} )|\Psi_{p}\rangle = 
\hat{Q}\hat{V}|\Psi_{p} \rangle - \hat{Q}[\langle p | \hat{V}\hat{Q}|\Psi_{p} \rangle] \hat{Q}|\Psi_{p}\rangle, 
\end{equation*}
noting that  $[\langle p | \hat{V}\hat{Q}|\Psi_{p} \rangle] = [\langle p | \hat{V} |\Psi_{p} \rangle]$, which
is just a number, and that $Q$ is idempotent, this can be rewritten
\begin{equation}
( \epsilon_{p}- \hat{H}_{0} )|\Psi_{p}\rangle = 
\hat{Q}\hat{V}|\Psi_{p} \rangle -  \hat{Q}|\Psi_{p}\rangle[\langle p | \hat{V} |\Psi_{p} \rangle]. 
\label{eqn:Q_on_both_sides_singlestate}
\end{equation}
which is a Bloch equation. \\

\noindent As mentioned above, the case where $\mathrm{p}$ contains more than one state is less straightforward, and there
a number of approaches to dealing with it (MS-CASPT2, XMS-CASPT2, GVVPT2 etc., ).  To explain the issues associated with the
degenerate multistate case, and to illustrate the different in motivation and structure of these approaches,
the above expressions are now rewritten using an explicit basis for spaces $\mathrm{p}$ and $\mathrm{q}$:
\begin{equation*}
\hat{P} =\sum_{p} | p \rangle \langle p |  \text{\ \ \ \ \ }
\hat{Q} =\sum_{q} | q \rangle \langle q | ,
\end{equation*}
\begin{equation*}
\hat{P}|\Psi_{p}\rangle =\sum_{p}\sum_{l} | p \rangle \langle p | l \rangle X_{l}^{p}
\text{ \ \ \ \ \ \ }
\hat{Q}|\Psi_{p}\rangle =\sum_{q}\sum_{r} | q \rangle \langle q | r \rangle T_{r}^{p},
\end{equation*}
where $X_{l}^{p}$ and $T_{r}^{p}$ are coefficients. The $X_{l}^{p}$ may be interpreted
as the coefficients describing how the perturbation acts to mix together the
states within the reference space, $\mathrm{p}$. If the states within the reference 
space are well seperated energetically, then it is reasonable to assume that this
mixing is small, and $X_{l}^{p} \rightarrow \delta_{lp} $, however, this is not the case,
by definition, in the vicinity of conical intersections.\\

\noindent In the following initial discussion of approaches to this problem it shall be assumed that 
the sets of $\{|l\rangle\}$ and $\{|r\rangle\}$ being summed over are identitical to the respective sets
$\{|p\rangle\}$ and $\{|q\rangle\}$. This assumption is not valid in the case of internally contracted 
methods, which will be discussed in due course.\\

\noindent Rewriting (\ref{eqn:P_on_EH}) with this basis gives
\begin{equation*}
\sum_{mnl}|m\rangle \langle m | (\epsilon_{p} - \hat{H}_{0})|n\rangle \langle n | l \rangle X_{l}^{p}
+ \sum_{mqr}|m\rangle \langle m | (\epsilon_{p} - \hat{H}_{0})|q\rangle \langle q | r \rangle T_{r}^{p}
\end{equation*}
\begin{equation*}
\sum_{mnl}|m\rangle \langle m | (\epsilon_{p} - \hat{H}_{0})   | l \rangle X_{l}^{p}
+ \sum_{mqr}|m\rangle \langle m | (\epsilon_{p} - \hat{H}_{0}) | r \rangle T_{r}^{p}
\label{eqn:P_on_EH_explicit_bas}
\end{equation*}
Multiplying from the left by state $\langle\Psi_{p}| = \sum_{k}\langle k | X_{k}^{p}$ yields
\begin{equation}
\sum_{kmnl}X^{p\dagger}\langle k | m\rangle \langle m | (\epsilon_{p} - \hat{H}_{0})   | l \rangle X_{l}^{p}
+ \sum_{kmqr}X^{p\dagger}\langle k | (\epsilon_{p} - \hat{H}_{0}) | r \rangle T_{r}^{p}
\end{equation}
\begin{equation}
=  \sum_{kl}X^{p\dagger}\langle k | (\epsilon_{p} - \hat{H}_{0}) | l \rangle X_{l}^{p}
\label{eqn:P_on_EH_ms_nondiag}
\end{equation}
Substituing this back into (\ref{eqn:multistate_pt_energy}) results in 
\begin{equation}
\sum_{kl}X^{p\dagger}\langle k | (\epsilon_{p} - \hat{H}_{0}) | l \rangle X_{l}^{p}
\label{eqn:multistate_PT_energy_nondiag}
\end{equation}
The useful expression (\ref{eqn:singlestate_pt_energy}) for the perturbation energy, $E_{p}$, 
may be obtained because (\label{eqn:P_on_EH_ms_nondiag}) vanishes. In state specific MS-CASPT2
the expression is made to vanish through use of the assumption that $\hat{H}_{0}$ has eigenvectors
$\{|p\}$ with eigenvalues $\epsilon_{p}$. Considering that the
$\{|p\rangle\}$ can easily be chosen such that this is the case this assumption is entirely resonable.
However, that $\hat{H}_{0}$ is diagonal in $\{|p\rangle\}$ is no guarantee that derivatives
of $\hat{H}_{0}$ are diagonal in $\{|p\rangle\}$, hence terms such as 
\begin{equation}
\frac{\delta}{\delta \mathbf{R} }\sum_{kl}X^{p\dagger}\langle k | (\epsilon_{p} - \hat{H}_{0}) | l \rangle X_{l}^{p},
\label{eqn:multistate_PT_energy_nondiag_deriv}
\end{equation}
where $\mathbf{R}$ is some perturbing parameter, may not vanish. Whilst the basic expressions
for the peturbation energy may not appear to require consideration of such terms, they are highly relevant
in developing computational methods for calculating the perturbation energies, and in obtaining expressions
for derivative properties.\\

\noindent In XMS-CASPT2 the zeroth order Hamiltonian is
\begin{equation}
\hat{H}_{0,XMS} = \hat{P}\hat{f}\hat{P} +  \hat{Q}\hat{f}\hat{Q},
\end{equation}
where $\hat{f}$ is the state averaged Fock operator defined by
\begin{equation}
\hat{f} = \sum_{rs}f_{rs} = \sum_{rs}
\hat{a}^{\dagger}_{r}\hat{a}_{s}
\Bigg{[} h_{rs} 
+ \sum_{ij}d_{ij}
\frac{1}{2}( 2J^{ij}_{rs}-K^{ij}_{rs})  \Bigg{]}
\end{equation}
where, $J^{ij}_{rs} = (ij|rs)$ and $K^{ij}_{rs} = (ir|js)$ are the two electron integrals, and $d_{ij}$ are elements of the reduced, one-electron,
state-averaged density matrix. The states $\{|p\rangle\}$ are unitary transformed into a set $\{|\tilde{p}\rangle\}$, 
which are eigenvectors of $\hat{H}_{0,XMS}$ with eigenvalues $\tilde{\epsilon}_{p}$;
\begin{equation}
|\tilde{p} = \sum_{l}|l\rangle
\end{equation}
The importance of state averaging will become apparent in later stages of the derivation, 
where it ensures that the elements $f_{ij}$ are invariant under rotations
between states. From now on, the $XMS$ subscript on $\hat{H}_{0,XMS}$ will be omitted.\\ 

\noindent  A final comment before proceeding is that a unitary transformation of
$\mathrm{p}$ should have no impact on the final results, i.e., whether $|\Psi_{p} \rangle$
is considered to be a perturbation of state $|p\rangle$ or $|\tilde{p}\rangle$ should
not make any difference. \\

(\ref{eqn:Q_on_both_sides_singlestate}),
If there is more than one state in 
\begin{equation*}
  \sum_{kl}X^{p\dagger}\langle m | (\epsilon_{p} - \hat{H}_{0}) | l \rangle X_{l}^{p}
+ \sum_{kr}X^{p\dagger}\langle m | (\epsilon_{p} - \hat{H}_{0}) | r \rangle T_{r}^{p}
\end{equation*}

\end{document}
