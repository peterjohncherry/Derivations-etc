\documentclass[12pt]{article}
\usepackage[utf8x]{inputenc}
\usepackage[english]{babel}
\usepackage[T1]{fontenc}
\usepackage{amsmath}
\usepackage{amssymb}
\usepackage{textcomp}
\usepackage{array}
\usepackage{pdfpages}
\usepackage{geometry}
 \geometry{
 a4paper,
 total={170mm,257mm},
 left=15mm,
 top=15mm,
 bottom=15mm,
 right=15mm,
 }

\usepackage{cite}
\linespread{1.3}

\begin{document}
\section{ Supplementary material }
Begin by defining a space $\mathrm{s}$, split into a reference, $\mathrm{p}$, and external, $\mathbf{q}$, space;
\begin{equation}
\mathrm{s} = \mathrm{p} + \mathrm{s} = \{ |p\rangle \} \cup \{|q\rangle \}
\end{equation}
Now define projection operators
\begin{equation}
\hat{P} =\sum_{p} |p\rangle \langle p | 
\text{ \ \ \ \ \ \ \ \ \ }
\hat{Q} =\sum_{q} |q\rangle \langle q | 
\end{equation}
\begin{equation}
\hat{P}+\hat{Q} = \hat{I}
\text{\ \ \ \ \ \ \ \ \ }
\hat{P}-\hat{I} = \hat{Q}
\end{equation}
Then have zeroth order Hamiltonian, $\hat{H}_{0}$, and perturbation, $\hat{V}$. The spaces $\mathrm{p}$ and $\mathrm{q}$ are such that
\begin{equation}
\langle p | \hat{H}_{0} | q \rangle  = 
\langle q | \hat{H}_{0} | p \rangle  = 
\langle p | \hat{V} | p \rangle  =  0 \text{ \ \ \ } \forall \text{ \ }p\text{,\ }q  . 
\label{eqn:space_props}
\end{equation}
Write two equations for the zeroth order, $\hat{H}_{0}$, and full, $\hat{H} = \hat{H}_{0} + \hat{V}$, Hamiltonians:
\begin{equation}
\hat{H}_{0} | p \rangle  = \epsilon_{p} |p \rangle
\end{equation}
\begin{equation}
\hat{H}|\Psi_{p} \rangle =( \hat{H}_{0}+\hat{V} )| \Psi_{p} \rangle  = (\epsilon_{p}+E_{p})|\Psi_{p} \rangle
\end{equation}
Here, $|\Psi_{p}\rangle$, is being treated as though it were a perturbation of state $|p\rangle$. Initially, 
assume that there is only one state in the space $\mathrm{p}$. It is now more convenient to write this as
\begin{equation}
(\epsilon_{p}-\hat{H}_{0} )| \Psi_{p} \rangle  = (\hat{V}-E_{p})|\Psi_{p} \rangle
\end{equation}
The wavefunction $|\Psi_{p} \rangle$ may be expanded in $\mathrm{p}$ and $\mathrm{q}$;
\begin{equation}
| \Psi_{p} \rangle  = \sum_{p} C_{p}|p\rangle + \sum_{q} T_{q}|q\rangle.
\end{equation}
Hence,
\begin{equation}
\hat{P} | \Psi_{p} \rangle  = \sum_{p} C_{p}|p\rangle
\text{ \ \ \ \  and \ \ \ \ }
\hat{Q} | \Psi_{p} \rangle  = \sum_{q} T_{q}|q\rangle .
\end{equation}
This, combined with the identities associated with the projectors $\hat{P}$ and $\hat{Q}$, and the
attributes of the space, $\mathrm{s}$, stated in (\ref{eqn:space_props}) leads to the following statement
\begin{equation}
\hat{P}(\epsilon_{p} - \hat{H}_{0})(\hat{P}+\hat{Q})| \Psi_{p} \rangle  =  \hat{P}(\hat{V}-E_{p})(\hat{P}+\hat{Q})|\Psi_{p} \rangle .
\label{eqn:P_on_both sides}
\end{equation}
Taking each side seperately:
\begin{equation}
\hat{P}(\epsilon_{p} - \hat{H}_{0})(\hat{P}+\hat{Q})| \Psi_{p} \rangle  = 0  ,
\label{eqn:P_on_EH}
\end{equation}
\begin{equation}
\hat{P}(\hat{V}-E_{p})(\hat{P}+\hat{Q})|\Psi_{p} \rangle = \hat{P}\hat{V}\hat{Q}|\Psi_{p} \rangle + \hat{Q}E_{p}\hat{Q}|\Psi_{p}\rangle .
\label{eqn:P_on_VE}
\end{equation}
Substutiting (\ref{eqn:P_on_EH}) and (\ref{eqn:P_on_VE}) back into (\ref{eqn:P_on_both sides}) and rearranging we obtain
\begin{equation}
\hat{P}\hat{V}\hat{Q}|\Psi_{p} \rangle =  \hat{P}E_{p}\hat{P}|\Psi_{p}\rangle .
\label{eqn:pt_energy}
\end{equation}
Assuming the members of $\mathrm{p}$ are orthonormal and taking advantage of the consequent idempotency of $\hat{P}$ leads to
\begin{equation}
\hat{P}E_{p}\hat{P}|\Psi_{p}\rangle = E_{p}\hat{P}|\Psi_{p} \rangle ,
\end{equation}
hence
\begin{equation}
\sum_{p'} \langle p' | E_{p}\hat{P}|\Psi_{p} \rangle  =  \sum_{p'} \langle p' | \hat{V}\hat{Q}|\Psi_{p} \rangle.
\label{eqn:multistate_pt_energy}
\end{equation}
Whilst the primary intention of the MRPTool module is to tackle multistate, highly degenerate cases, it helpful\footnote{for me, at least} 
to first consider the singlestate case, i.e., where the dimenation of $\mathrm{p}$ is 1. 
If intermediate normalization, $\langle p | \Psi_{p} \rangle = 1$, is assumed then 
\begin{equation}
\hat{P}E_{p}\hat{P}|\Psi_{p}\rangle = E_{p},
\end{equation}
which leads to 
\begin{equation}
E_{p} = \langle p | \hat{V}\hat{Q}|\Psi_{p} \rangle.
\label{eqn:singlestate_pt_energy}
\end{equation}
If in (\ref{eqn:P_on_both sides}) $\hat{Q}$ instead of $\hat{P}$ was applied from the left the following expressions would result:
\begin{equation}
\hat{Q}(\hat{V}-E_{p})(\hat{P}+\hat{Q})|\Psi_{p} \rangle = \hat{Q}\hat{V}|\Psi_{p} \rangle - \hat{Q}E_{p}\hat{Q}|\Psi_{p}\rangle ,
\label{eqn:Q_on_VE}
\end{equation}
\begin{equation}
\hat{Q}(\epsilon - \hat{H}_{0})(\hat{P}+\hat{Q})| \Psi_{p} \rangle  = \hat{Q}\epsilon_{p}\hat{Q}|\Psi\rangle- \hat{Q}\hat{H}_{0}\hat{Q}|\Psi_{p}\rangle.
\end{equation}
However, noting that the RHS of (\ref{eqn:P_on_EH}) is $0$ it is possible to write
\begin{equation*}
\hat{Q}(\epsilon_{p} - \hat{H}_{0})(\hat{P}+\hat{Q})| \Psi_{p} \rangle =
(\hat{P} + \hat{Q})(\epsilon_{p} - \hat{H}_{0})(\hat{P}+\hat{Q})| \Psi_{p} \rangle =
(\epsilon_{p} - \hat{H}_{0})| \Psi_{p} \rangle ,
\end{equation*}
hence
\begin{equation}
(\epsilon - \hat{H}_{0})| \Psi_{p} \rangle  = \hat{Q}\epsilon_{p}\hat{Q}|\Psi\rangle- \hat{Q}\hat{H}_{0}\hat{Q}|\Psi_{p}\rangle.
\label{eqn:Q_on_EH} 
\end{equation}
Equating the RHS's of (\ref{eqn:Q_on_EH}) and (\ref{eqn:Q_on_VE})
 \begin{equation*}
(\epsilon_{p} - \hat{H}_{0})|\Psi_{p}\rangle
= \hat{Q}\hat{V}|\Psi_{p} \rangle - \hat{Q}E_{p}\hat{Q}|\Psi_{p}\rangle .
\end{equation*}
If there is only one state in $\mathrm{p}$, then it is possible to substitute in from  (\ref{eqn:singlestate_pt_energy}) to obtain
\begin{equation*}
( \epsilon_{p}- \hat{H}_{0} )|\Psi_{p}\rangle = 
\hat{Q}\hat{V}|\Psi_{p} \rangle - \hat{Q}[\langle p | \hat{V}\hat{Q}|\Psi_{p} \rangle] \hat{Q}|\Psi_{p}\rangle, 
\end{equation*}
noting that  $[\langle p | \hat{V}\hat{Q}|\Psi_{p} \rangle] = [\langle p | \hat{V} |\Psi_{p} \rangle]$, which
is just a number, and that $Q$ is idempotent, this can be rewritten
\begin{equation}
( \epsilon_{p}- \hat{H}_{0} )|\Psi_{p}\rangle = 
\hat{Q}\hat{V}|\Psi_{p} \rangle -  \hat{Q}|\Psi_{p}\rangle[\langle p | \hat{V} |\Psi_{p} \rangle]. 
\label{eqn:Q_on_both_sides_singlestate}
\end{equation}
It is useful to define a "wave operator" $\hat{\Omega}^{p}$ which can be used to obtain the perturbed state, $|\Psi_{p}\rangle$,
from a state $\{ |p\rangle \}$ within the reference space $\mathrm{p}$;
\begin{equation}
|\Psi_{p} \rangle = \hat{\Omega}_{p}|p\rangle .
\end{equation}
This may be used to rewrite (\ref{eqn:Q_on_both_sides_singlestate}) as 
\begin{equation}
( \epsilon_{p}- \hat{H}_{0} )\hat{\Omega}^{p}|p\rangle = 
\hat{Q}\hat{V}\hat{\Omega}^{p}|p \rangle -  \hat{Q}\hat{\Omega}^{p}|p\rangle[\langle p | \hat{V}\hat{\Omega}^{p} |p \rangle]. 
\label{eqn:Bloch_singlestate}
\end{equation}
CASPT2 and related theories discussed in the following can be thought of as techniques for finding approximate methods for finding $\Omega_{p}$. 

\section{ CASPT2 }
\noindent The wave operator can be written as a series expansion;
\begin{equation}
\hat{\Omega}^{p} = \hat{\Omega}^{p,0}+\hat{\Omega}^{p,1}+\hat{\Omega}^{p,2}+....
\end{equation} 
where 
\begin{equation}
\hat{\Omega}^{p,0}|p\rangle = |p^{0}\rangle = |p\rangle \text{, \ \  }
\hat{\Omega}^{p,1}|p\rangle = |p^{1}\rangle \text{, \ \ }
\hat{\Omega}^{p,2}|p\rangle = |p^{2}\rangle \text{, \  etc.. }
\label{eqn:wave_op_series}
\end{equation} 
Where  $\langle p^{i} | p^{j} \rangle=  \delta_{ij}$,  hence $(\hat{Q}\hat{\Omega}^{p,0}) = 0$.
Expanding out the energy in a similar manner,
substituting the above expansion into (\ref{eqn:Block_singlestate}) and equating first order terms leads to the first order
wave equation,
\begin{equation}
(\epsilon_{p} - \hat{H}_{0} )\Omega^{p,1}|p\rangle = \hat{Q}\hat{H}|p\rangle ,
\label{eqn:Bloch_singlestate_firstorder}
\end{equation}
which makes use of the fact that $ \hat{Q}\hat{V}|p\rangle= \hat{Q}\hat{H}|p\rangle$. A similar logic may be
used to obtain an expression for the second order energy;
\begin{equation}
E_{p}^{(2)} = E_{p}^{(1)} + \langle p | \hat{H} \hat{\Omega}^{p,1}| p \rangle .
\end{equation}
Provided the above orthogonality constraints are met there is considerable freedom in choosing the form of the terms in the
series $(\ref{eqn:wave_op_series})$.  A natural choice is to have $\hat{\Omega}^{p,1}$, be the contribution
to $|\Psi_{p}\rangle$ from single and doubly excitated states, $\hat{\Omega}^{p,2}$,  the contribution from triply
and quadruply excitated states, and so on:
\begin{equation*}
\hat{\Omega}^{p,0} = \sum_{m} |m \rangle \langle m |   \text{, \ \ \  } m \in \mathrm{p}
\end{equation*}
\begin{equation*}
\hat{\Omega}^{p,1} = \sum_{r}| r \rangle \langle p | T^{p}_{r} \text{ \ \ \ } r \in \mathrm{q}^{sd} 
\end{equation*}
\begin{equation*}
\hat{\Omega}^{p,2} = \sum_{x}| x \rangle \langle p | T^{p}_{x} \text{ \ \ \ } x \in \mathrm{q}^{tq}  \text{, \  etc.. }.
\end{equation*}
The above makes use of the fact that the space, $q$, may be split up into mutually orthogonal 
subspaces corresponding to singly and doubly excitated states, triply and quadruply excited states, etc..,
\begin{equation}
\mathrm{q} = \mathrm{q}^{sd} + \mathrm{q}^{tq} + ...
\end{equation}
The $T^{p}_{r}$ are amplitude coefficients indicating how much $|r\rangle \in q^{(st)}$
contributes to $|\Psi_{p}\rangle$.\\

\noindent In the internally contracted framework projection into these excited spaces is achieved via excitation operators, e.g., $\hat{E}_{\omega}$,
which excite the appropriate number of electrons from the closed or active orbitals specified by $\omega$; 
\begin{equation}
\hat{E}^{sd}_{\omega}|p\rangle = |r_{p}\rangle  \text{ \ \ \ \ } |r\rangle \in q^{sd} .
\end{equation}
Note that the state onto which the operator $\hat{E}^{sd}_{\omega}$ projects is dependent on the
the state $|p\rangle $  upon which it acts. So in principal, it is possible to replace the index denoting 
orbital excitations with a subscripted index, $\omega_{p}$, where $\omega_{p}$ is the state onto which
the action of $\hat{E}_{\omega}$ on $|p\rangle$ projects, i.e.,
\begin{equation}
\hat{E}_{\omega} |p\rangle \rightarrow 
\hat{E}_{r_{p}} |p\rangle = | r_{p} \rangle \langle p | p \rangle .
\end{equation}
Though needlessly prolix when the subspace $\mathrm{p}$ contains only one state, this notation will
come in use later on when dealing with the multistate case. However, until that point, the subscript on $p$
shall be omitted.\\

\noindent Rewriting (\ref{eqn:Bloch_singlestate_firstorder}) using these definitions gives
\begin{equation}
\sum_{r'}(\epsilon_{p} - \hat{H}_{0} )T_{r'}^{st} \Omega^{p,1}|p\rangle = \sum_{r}\hat{Q}^{st}_{r}\hat{H}|p\rangle .
\label{eqn:Bloch_singlestate_firstorder_st}
\end{equation}
\begin{equation*}
\sum_{r}^{r\in q^{st}} (\epsilon_{p} - \hat{H}_{0} )T_{r}^{p} \hat{E}_{r} | p \rangle 
= 
\sum_{r}^{r\in q^{st}} |r \rangle \langle r | \hat{H}|p\rangle .
\end{equation*}
Rearranging and left multiplying by $\langle r | \in q^{st}$ gives  
\begin{equation*}
\langle r' | \sum_{r}^{r\in q^{st}} (\epsilon_{p} - \hat{H}_{0} )T_{r}^{p} | r \rangle 
-
\langle r' | \sum_{r}^{r\in q^{st}} | r \rangle \langle r | \hat{H} |p\rangle = 0
\end{equation*}
\begin{equation}
\langle r' | \sum_{r}^{r\in q^{st}} (\epsilon_{p} - \hat{H}_{0} )T_{r}^{p} | r \rangle 
-
\langle r' | \hat{H} |p\rangle = 0
\label{eqn:Amplitude_eqn}
\end{equation}
%Generally speaking, the projection operators, e.g., $Q^{sd}_{p}$,
%are replaced with excitation operators, e.g.,
%\begin{equation}
%\hat{Q}^{sd} = \sum_{ij}\sum_{kl}\hat{a}^{\dagger}_{i}\hat{a}^{\dagger}_{j}\hat{a}_{k}\hat{a}_{l}.
%\end{equation}
%The ranges of the summations over indexes, $k,l$, and ,$i,j$, are defined such that this corresponds the appropriate projection.
%This leads to
%\begin{equation}
% \sum_{r}^{r\in q^{st}}
%\langle p | \hat{E}^{\dagger}_{r'}(\epsilon_{p} - \hat{H}_{0} )T_{r}^{p}\hat{E}_{r}| p \rangle 
%-
%\langle p | \hat{E}^{\dagger}_{r'}\hat{H} |p\rangle = 0
%\label{eqn:Hylleraas_st}
%\end{equation}
The derivative with respect $T$ should vanish, so at convergence,
\begin{equation}
\sum_{r}^{r\in q^{st}} \langle r' |(\epsilon_{p} - \hat{H}_{0} )T_{r}^{p} | r \rangle = G = 0 
\label{eqn:caspt2_residual}
\end{equation}
Using the definition of the functional derivative;
\begin{equation*}
\frac{G[T_{r}^{p} + \delta T ] - G[T_{r}^{p}]}{ \delta T_{r}^{p} } = 0,
\end{equation*}
\begin{equation}
\frac{\delta G[T^{p}_{r}]} { \delta T^{p}_{r}}  = 
\langle r |(\epsilon_{p} - \hat{H}_{0} )| r \rangle.
\end{equation}
This can now be used to obtain the T update equation
\begin{equation}
\Delta T_{r} = \Bigg{(}\frac{\delta G[T^{p}_{r}]} { \delta T^{p}_{r}}\Bigg{)}^{-1}G[T^{p}_{r}]  = 
\frac{\sum_{r}^{r\in q^{st}} \langle r' |(\epsilon_{p} - \hat{H}_{0} )T_{r}^{p} | r \rangle} 
{\langle r' |(\epsilon_{p} - \hat{H}_{0} )| r' \rangle}.
\label{eqn:T_update}
\end{equation}
This expression is iterated over until the threshold is reached. 

\section{ Multistate cases }
\noindent As mentioned above, the case where $\mathrm{p}$ contains more than one state is less straightforward, and there
a number of approaches to dealing with it (MS-CASPT2, XMS-CASPT2, GVVPT2 etc., ).  To explain the issues associated with the
degenerate multistate case, and to illustrate the different in motivation and structure of these approaches,
the above expressions are now rewritten using an explicit basis for spaces $\mathrm{p}$ and $\mathrm{q}$:
\begin{equation*}
\hat{P} =\sum_{p} | p \rangle \langle p |  \text{\ \ \ \ \ }
\hat{Q} =\sum_{q} | q \rangle \langle q | ,
\end{equation*}
\begin{equation*}
\hat{P}|\Psi_{p}\rangle =\sum_{p}\sum_{l} | p \rangle \langle p | l \rangle X_{l}^{p}
\text{ \ \ \ \ \ \ }
\hat{Q}|\Psi_{p}\rangle =\sum_{q}\sum_{r} | q \rangle \langle q | r \rangle T_{r}^{p},
\end{equation*}
where $X_{l}^{p}$ and $T_{r}^{p}$ are coefficients. The $X_{l}^{p}$ may be interpreted
as the coefficients describing how the perturbation acts to mix together the
states within the reference space, $\mathrm{p}$. If the states within the reference 
space are well seperated energetically, then it is reasonable to assume that this
mixing is small, and $X_{l}^{p} \rightarrow \delta_{lp} $, however, this is not the case,
by definition, in the vicinity of conical intersections.\\

\noindent In the following initial discussion of approaches to this problem it shall be assumed that 
the sets of $\{|l\rangle\}$ and $\{|r\rangle\}$ being summed over are identitical to the respective sets
$\{|p\rangle\}$ and $\{|q\rangle\}$. This assumption is not valid in the case of internally contracted 
methods, which will be discussed in due course.\\

\noindent Rewriting (\ref{eqn:P_on_EH}) with this basis gives
\begin{equation*}
\sum_{mnl}|m\rangle \langle m | (\epsilon_{p} - \hat{H}_{0})|n\rangle \langle n | l \rangle X_{l}^{p}
+ \sum_{mqr}|m\rangle \langle m | (\epsilon_{p} - \hat{H}_{0})|q\rangle \langle q | r \rangle T_{r}^{p}
\end{equation*}
\begin{equation*}
\sum_{mnl}|m\rangle \langle m | (\epsilon_{p} - \hat{H}_{0})   | l \rangle X_{l}^{p}
+ \sum_{mqr}|m\rangle \langle m | (\epsilon_{p} - \hat{H}_{0}) | r \rangle T_{r}^{p}
\label{eqn:P_on_EH_explicit_bas}
\end{equation*}
Multiplying from the left by state $\langle\Psi_{p}| = \sum_{k}\langle k | X_{k}^{p}$ yields
\begin{equation*}
\sum_{kmnl}X^{p\dagger}_{k}\langle k | m\rangle \langle m | (\epsilon_{p} - \hat{H}_{0})   | l \rangle X_{l}^{p}
+ \sum_{kmqr}X^{p\dagger}_{k}\langle k | (\epsilon_{p} - \hat{H}_{0}) | r \rangle T_{r}^{p}
\end{equation*}
\begin{equation}
=  \sum_{kl}X^{p\dagger}_{k}\langle k | (\epsilon_{p} - \hat{H}_{0}) | l \rangle X_{l}^{p}
\label{eqn:P_on_EH_ms_nondiag}
\end{equation}
%Substituing this back into (\ref{eqn:multistate_pt_energy}) results in 
%\begin{equation}
%\sum_{kl}X^{p\dagger}_{k}\langle k | (\epsilon_{p} - \hat{H}_{0}) | l \rangle X_{l}^{p}
%\label{eqn:multistate_PT_energy_nondiag}
%\end{equation}
The useful expression (\ref{eqn:singlestate_pt_energy}) for the perturbation energy, $E_{p}$, 
may be obtained because the RHS of (\ref{eqn:P_on_EH}); the single state analogue of (\ref{eqn:P_on_EH_ms_nondiag}), vanishes.
Unfortunately, this does not happen here, a fact which will later prove significant.
There is more than one way of handling this issue, but to summarize them properly it is
worth first discussing the generalized Bloch equation.

\subsection{ Generalized Bloch Equation } 
Consider the matrix equation,
\begin{equation}
\mathbf{H}\boldsymbol{\Psi} =
\begin{bmatrix}
 \mathbf{H_{PP}} & \mathbf{H_{PQ}} \\ 
 \mathbf{H_{QP}} & \mathbf{H_{QQ}} \\ 
\end{bmatrix} 
\begin{bmatrix}
 \boldsymbol{\Psi}_{P} \\ 
 \boldsymbol{\Psi}_{Q} \\ 
\end{bmatrix} 
= E 
\begin{bmatrix}
 \boldsymbol{\Psi}_{P} \\ 
 \boldsymbol{\Psi}_{Q} \\ 
\end{bmatrix} 
\end{equation}
Now introduce a transformation operator, $\mathbf{S}$, which can be used to obtain the full 
vector, $\boldsymbol{\Psi}$, from just the component, $\boldsymbol{\Psi}_{P}$ within the $\mathrm{p}$ subspace, i.e.,
\begin{equation}
\mathbf{S}(\mathbf{T})\boldsymbol{\Psi}_{0}= 
\begin{bmatrix}
\mathbf{I}_{PP} & 0 \\ 
\mathbf{T} & \mathbf{I}_{QQ} \\ 
\end{bmatrix}
\begin{bmatrix}
\boldsymbol{\Psi}_{P} \\ 
0 \\ 
\end{bmatrix}=  
\begin{bmatrix}
\boldsymbol{\Psi}_{P} \\ 
\mathbf{T}\boldsymbol{\Psi}_{P} \\ 
\end{bmatrix} 
= \boldsymbol{\Psi}
\end{equation}
The matrix $\mathbf{S}(\mathbf{T})$ has useful property
\begin{equation}
\mathbf{S}(\mathbf{T}_{1})+ \mathbf{S}(\mathbf{T}_{2})=\mathbf{S}(\mathbf{T}_{1}+\mathbf{T}_{2}),
\end{equation}
which has the corollary  $\mathbf{S}(\mathbf{T})^{-1}=\mathbf{S}(-\mathbf{T})$ . Noting this it is possible to write
\begin{equation}
\mathbf{S}(\mathbf{T})^{-1} \mathbf{H} \mathbf{S}(-\mathbf{T})\boldsymbol{\Psi}_{0} = 
\end{equation}
\begin{equation}
\begin{bmatrix}
\mathbf{I}_{PP} & \mathbf{0} \\ 
-\mathbf{T} & \mathbf{I}_{QQ} \\ 
\end{bmatrix}
\begin{bmatrix}
 \mathbf{H}_{PP} & \mathbf{H}_{PQ} \\ 
 \mathbf{H}_{QP} & \mathbf{H}_{QQ} \\ 
\end{bmatrix} 
\begin{bmatrix}
\mathbf{I}_{PP} & \mathbf{0} \\ 
\mathbf{T} & \mathbf{I}_{QQ} \\ 
\end{bmatrix}
\begin{bmatrix}
 \boldsymbol{\Psi}_{P} \\ 
 \mathbf{0}\\ 
\end{bmatrix} 
= E 
\begin{bmatrix}
 \boldsymbol{\Psi}_{P} \\ 
 \mathbf{0}\\ 
\end{bmatrix} 
\label{eqn:TransEvalEqn}
\end{equation}
The blocks of the transformed matrix $\mathbf{\tilde{H}} = \mathbf{S}(\mathbf{T})^{-1} \mathbf{H} \mathbf{S}(-\mathbf{T})$ are
\begin{equation}
\mathbf{\tilde{H}}_{PP} = \mathbf{H}_{PP} + \mathbf{H}_{PQ}\mathbf{T}
\label{eqn:HPP}
\end{equation} 
\begin{equation}
\mathbf{\tilde{H}}_{PQ} = \mathbf{H}_{PQ}
\end{equation} 
\begin{equation}
\mathbf{\tilde{H}}_{QP} = \mathbf{H}_{QP} + \mathbf{H}_{QQ}\mathbf{T} - \mathbf{\mathbf{H}}_{PP}\mathbf{T} + \mathbf{T}\mathbf{H}_{PQ}\mathbf{T}
\label{eqn:HQP}
\end{equation} 
\begin{equation}
\mathbf{\tilde{H}}_{QQ} = \mathbf{H}_{QQ} -\mathbf{T}\mathbf{H}_{PQ}
\end{equation} 

\noindent The expression (\ref{eqn:HPP}) defines the effective Bloch Hamiltonian, and can be directly connected to the Bloch equation
specified in (\ref{eqn:Bloch_singlestate}). Requiring that $\mathbf{T}$ is such that (\ref{eqn:HQP}) is zero will decouple the
$H_{PP}$ block rest of the blocks of the transformed matrix, enabling the obtain eigenvalues associated with the $PP$ block to be obtained
seperately from those associated with the rest of the matrix;
\begin{equation}
\mathbf{\tilde{H}}
\begin{bmatrix}
 \boldsymbol{\psi_{P}} \\ 
 \mathbf{0} \\ 
\end{bmatrix}
=
\begin{bmatrix}
 \mathbf{\tilde{H}_{PP}} & \mathbf{\tilde{H}_{PQ}} \\ 
 0                       & \mathbf{\tilde{H}_{QQ}} \\ 
\end{bmatrix} 
\begin{bmatrix}
 \boldsymbol{\Psi_{P}} \\ 
 \mathbf{0} \\ 
\end{bmatrix} 
= E 
\begin{bmatrix}
\boldsymbol{\Psi_{P}} \\ 
\mathbf{0} 
\end{bmatrix}.
\label{eqn:TransEvalEqnSimple}
\end{equation}
This can now be connected to the Bloch equation stated earlier by making use of the notation;
\begin{equation}
\mathbf{P} =  
\begin{bmatrix}
\mathbf{I} & \mathbf{0} \\ 
\mathbf{0} & \mathbf{0} 
\end{bmatrix},
\text{ \ \ \ }
\mathbf{Q} =  
\begin{bmatrix}
\mathbf{0} & \mathbf{0} \\ 
\mathbf{0} & \mathbf{I} 
\end{bmatrix},
\text{ \ \ \ }
\boldsymbol{\Omega} =  
\begin{bmatrix}
\mathbf{I} & \mathbf{0} \\ 
\mathbf{F} & \mathbf{0} 
\end{bmatrix},
\text{ \ \ \ }
\boldsymbol{\Psi}^{0} =
\begin{bmatrix}
\boldsymbol{\Psi}_{P} \\
\mathbf{0} 
\end{bmatrix},
\end{equation}

Using this notation the Bloch Hamiltonian, $\mathbf{H}^{Bloch}$, may be written
\begin{equation}
\mathbf{H}^{Bloch} = \mathbf{P}\mathbf{H}\mathbf{\Omega}\mathbf{P}\mathbf{\Psi}.
\end{equation}
The generalized Bloch equation is  
\begin{equation}
\boldsymbol{\Omega}\mathbf{H}\boldsymbol{\Omega} =
\mathbf{H}\boldsymbol{\Omega}
\label{eqn:GeneralizedBloch}
\end{equation}
To see this more directly it can be written out using the matrices specified above;
\begin{equation*}
=
\begin{bmatrix}
\mathbf{1} & \mathbf{0} \\ 
\mathbf{T} & \mathbf{0} \\ 
\end{bmatrix}
\begin{bmatrix}
 \mathbf{H}_{PP} & \mathbf{H}_{PQ} \\ 
 \mathbf{H}_{QP} & \mathbf{H}_{QQ} \\ 
\end{bmatrix}
\begin{bmatrix}
\mathbf{1} & \mathbf{0} \\ 
\mathbf{T} & \mathbf{0} \\ 
\end{bmatrix}
=
\begin{bmatrix}
 \mathbf{H}_{PP} & \mathbf{H}_{PQ} \\ 
 \mathbf{H}_{QP} & \mathbf{H}_{QQ} \\ 
\end{bmatrix}
\begin{bmatrix}
\mathbf{1} & \mathbf{0} \\ 
\mathbf{T} & \mathbf{0} \\ 
\end{bmatrix}
\end{equation*}
\begin{equation*}
\begin{bmatrix}
 \mathbf{H}_{PP} + \mathbf{H}_{PQ}\mathbf{T}& \mathbf{0} \\ 
 \mathbf{T}\mathbf{H}_{PP} + \mathbf{T}\mathbf{H}_{PQ}\mathbf{T}& \mathbf{0} \\ 
\end{bmatrix}
=
\begin{bmatrix}
 \mathbf{H}_{PP} + \mathbf{H}_{PQ}\mathbf{T} & \mathbf{0} \\ 
 \mathbf{H}_{QP} + \mathbf{H}_{QQ}\mathbf{T} & \mathbf{0} \\ 
\end{bmatrix}
\label{eqn:BlochGenMat}
\end{equation*}
to see that the last two lines are equivalent note use the fact that $\mathbf{T}$ is such that $\tilde{\mathbf{H}}_{QP}$ 
as defined in (\ref{eqn:HQP}) vanishes. Rearranging (\ref{eqn:HQP}) yields 
\begin{equation*}
\mathbf{H}_{QP} + \mathbf{H}_{QQ}\mathbf{T} = \mathbf{\mathbf{H}}_{PP}\mathbf{T} + \mathbf{T}\mathbf{H}_{PQ}\mathbf{T}.
\label{eqn:Tcondition}
\end{equation*}
Hence the two sides of (\ref{eqn:BlochGenMat}) are equivalent. It should be noted that it is in the 
failure of the choice of $\mathbf{T}$ to satisfy the condition specified in (\ref{eqn:Tcondition}) 
that many errors originate.\\

\noindent Whilst (\ref{eqn:GeneralizedBloch}) is the perhaps the most generic form of the Bloch equation,
it is not the one most commonly seen in the CASPT2 literature, which is 
\begin{equation}
(E_{0} - \mathbf{H}_{0})\mathbf{\Omega} \boldsymbol{\Psi}_{0} =
 \mathbf{Q}\mathbf{V}\boldsymbol{\Omega}\boldsymbol{\Psi}_{0}- 
 \mathbf{Q}\boldsymbol{\Omega}\mathbf{P}\mathbf{V}\boldsymbol{\Omega}\boldsymbol{\Psi}_{0} 
\end{equation}
To obtain this expression introduce the notation
\begin{equation}
\mathbf{H}_{0}=
\begin{bmatrix}
\mathbf{H}_{PP} & \mathbf{0} \\
\mathbf{0} & \mathbf{H}_{QQ} \\
\end{bmatrix},
\text{ \ \ \ \ }
\mathbf{V} =
\begin{bmatrix}
\mathbf{0} & \mathbf{H}_{PQ} \\
\mathbf{H}_{QP} & \mathbf{0} \\
\end{bmatrix}.
\label{eqn:MatToAlgBloch}
\end{equation}
Where, 
\begin{equation}
\mathbf{H}_{0}\mathbf{\Psi}_{0} = \mathbf{E}^{(0)}\mathbf{\Psi}_{0}
\label{eqn:unperturbedHE}
\end{equation}
and 
\begin{equation}
(\mathbf{H}_{0}+\mathbf{V})\mathbf{\Psi} = (\mathbf{E}^{0}+\mathbf{E}^{'})\mathbf{\Psi}
\label{eqn:perturbedHE}
\end{equation}
Writing $\boldsymbol{\Psi}$ as $\boldsymbol{\Omega}\boldsymbol{\Psi}_{0}$, and using employing the generalized Bloch equation (\ref{eqn:GeneralizedBloch}) leads to
\begin{equation}
\boldsymbol{\Omega}(\mathbf{H}_{0}+\mathbf{V})\boldsymbol{\Omega}\mathbf{\Psi}_{0} =
\boldsymbol{\Omega}(\mathbf{E}^{0}+\mathbf{E}^{'})\boldsymbol{\Omega}\mathbf{\Psi}_{0}
\label{eqn:WWperturbedHE}
\end{equation}
and and expression for $\mathbf{E}^{'}$ (for those states within the $P$-block only);
\begin{equation}
\mathbf{E}^{'} = \boldsymbol{\Omega}\mathbf{P}\mathbf{V}\boldsymbol{\Omega}\mathbf{\Psi}_{0}.
\label{eqn:EprimeGen}
\end{equation}
Similarly, it is also possible to write
\begin{equation*}
(\mathbf{H}_{0}+\mathbf{V})\boldsymbol{\Omega}\mathbf{\Psi}_{0} =
(\mathbf{E}^{0}+\mathbf{E}^{'})\boldsymbol{\Omega}\mathbf{\Psi}_{0}
\end{equation*}
\begin{equation}
\rightarrow
(\mathbf{E}^{0}-\mathbf{H}_{0})\boldsymbol{\Omega}\mathbf{\Psi}_{0} =
 (\mathbf{V}-\mathbf{E}^{'})\boldsymbol{\Omega}\mathbf{\Psi}_{0} 
\label{eqn:WperturbedHE}
\end{equation}
Using the expression for $\mathbf{E}^{'}$ in  (\ref{eqn:EprimeGen}) gives
\begin{equation}
(\mathbf{E}^{0}-\mathbf{H}_{0})\boldsymbol{\Omega}\mathbf{\Psi}_{0} =
\mathbf{V}\boldsymbol{\Omega}\mathbf{\Psi}_{0} - 
\boldsymbol{\Omega}\mathbf{P}\mathbf{V}\boldsymbol{\Omega}\mathbf{\Psi}_{0}.
\label{eqn:BGenIntermediate}
\end{equation}
The RHS can now be multiplied by $(\mathbf{P}+\mathbf{Q})$,  and upon noting that
\begin{equation*}
\mathbf{P}\mathbf{V}\boldsymbol{\Omega}\mathbf{\Psi}_{0} = \mathbf{P}\boldsymbol{\Omega}\mathbf{P}\mathbf{V}\boldsymbol{\Omega}\mathbf{\Psi}_{0}.
\end{equation*}
this term can be cancelled, leaving
\begin{equation}
(\mathbf{E}^{0}-\mathbf{H}_{0})\boldsymbol{\Omega}\mathbf{\Psi}_{0} =
\mathbf{Q}\mathbf{V}\boldsymbol{\Omega}\mathbf{\Psi}_{0} - 
\mathbf{Q}\boldsymbol{\Omega}\mathbf{P}\mathbf{V}\boldsymbol{\Omega}\mathbf{\Psi}_{0},
\label{eqn:BGenMSCASPT2}
\end{equation}
which is the form in which the Bloch equation appears much of the perturbation theory literature.\\

\noindent It is important to note that whilst (\ref{eqn:BGenMSCASPT2}) is the principal equation,
for it to hold it (given the above definitions) it requires that $\mathbf{T}$ is such that
(\ref{eqn:HQP}) vanishes. Often, approximations are used in the defintion of the matrix components
which appear in (\ref{eqn:HQP}), and how these approximations impact the accuracy of methods based on
(\ref{eqn:BGenMSCASPT2}) is an important topic which we return to in due course.
 
\subsection{ Multistate CASTP2 and Extended Multistate CASPT2 }
 In state specific MS-CASPT2 the expression is made to vanish through use of the assumption that $\hat{H}_{0}$ has eigenvectors
$\{|p\}$ with eigenvalues $\epsilon_{p}$. Considering that the
$\{|p\rangle\}$ can easily be chosen such that this is the case this assumption is entirely resonable.
However, that $\hat{H}_{0}$ is diagonal in $\{|p\rangle\}$ is no guarantee that derivatives
of $\hat{H}_{0}$ are diagonal in $\{|p\rangle\}$, hence terms such as 
\begin{equation}
\frac{\delta}{\delta \mathbf{R} }\sum_{kl}X^{p\dagger}\langle k | (\epsilon_{p} - \hat{H}_{0}) | l \rangle X_{l}^{p},
\label{eqn:multistate_PT_energy_nondiag_deriv}
\end{equation}
where $\mathbf{R}$ is some perturbing parameter, may not vanish. Whilst the basic expressions
for the peturbation energy may not appear to require consideration of such terms, they are highly relevant
in developing computational methods for calculating the perturbation energies, and in obtaining expressions
for derivative properties. To consider the various approaches to this issue it is worth taking a 
brief detour to consider \\

\noindent In XMS-CASPT2 the zeroth order Hamiltonian is
\begin{equation}
\hat{H}_{0,XMS} = \hat{P}\hat{f}\hat{P} +  \hat{Q}\hat{f}\hat{Q},
\end{equation}
where $\hat{f}$ is the state averaged Fock operator defined by
\begin{equation}
\hat{f} = \sum_{rs}f_{rs} = \sum_{rs}
\hat{a}^{\dagger}_{r}\hat{a}_{s}
\Bigg{[} h_{rs} 
+ \sum_{ij}d_{ij}
\frac{1}{2}( 2J^{ij}_{rs}-K^{ij}_{rs})  \Bigg{]}
\end{equation}
where, $J^{ij}_{rs} = (ij|rs)$ and $K^{ij}_{rs} = (ir|js)$ are the two electron integrals, and $d_{ij}$ are elements of the reduced, one-electron,
state-averaged density matrix. The states $\{|p\rangle\}$ are unitary transformed into a set $\{|\tilde{p}\rangle\}$, 
which are eigenvectors of $\hat{H}_{0,XMS}$ with eigenvalues $\tilde{\epsilon}_{p}$;
\begin{equation}
|\tilde{p} = \sum_{l}|l\rangle U_{lp}
\end{equation}
The importance of state averaging will become apparent in later stages of the derivation, 
where it ensures that the elements $f_{ij}$ are invariant under rotations
between states. The $XMS$ subscript on $\hat{H}_{0,XMS}$ will be omitted hereafter.\\ 

\noindent  A final comment before proceeding is that a unitary transformation of
$\mathrm{p}$ should have no impact on the final results, i.e., whether $|\Psi_{p} \rangle$
is considered to be a perturbation of state $|p\rangle$ or $|\tilde{p}\rangle$ should
not make any difference. \\



(\ref{eqn:Q_on_both_sides_singlestate}),
If there is more than one state in  
\begin{equation*}
  \sum_{kl}X^{p\dagger}\langle m | (\epsilon_{p} - \hat{H}_{0}) | l \rangle X_{l}^{p}
+ \sum_{kr}X^{p\dagger}\langle m | (\epsilon_{p} - \hat{H}_{0}) | r \rangle T_{r}^{p}
\end{equation*}

\end{document}
