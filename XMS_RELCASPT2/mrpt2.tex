\documentclass[12pt]{article}
\usepackage[utf8x]{inputenc}
\usepackage[english]{babel}
\usepackage[T1]{fontenc}
\usepackage{color}
\usepackage{wrapfig}
\usepackage{amsmath}
\usepackage{amssymb}
\usepackage{textcomp}
\usepackage{array}
\usepackage{booktabs}
\usepackage{subfigure}
\usepackage[font=small,format=plain,labelfont=bf,up,textfont=it,up]{caption}
\usepackage{longtable}
\usepackage{calc}
\usepackage{setspace}
\usepackage{multirow}
\usepackage{hhline}
\usepackage{ifthen}
\usepackage{lscape}
\usepackage{relsize}
\usepackage{bbold}
\usepackage{mathtext}
\usepackage{pdfpages}
\usepackage{geometry}
 \geometry{
 a4paper,
 total={170mm,257mm},
 left=15mm,
 top=15mm,
 bottom=15mm,
 right=15mm,
 }

\usepackage{cite}
\linespread{1.3}

\begin{document}
Partitition the space up into three subspaces :
\begin{itemize}
\item (R) : Reference space; constructed from CASSCF wavefunctions corresponded to the states of interest.
\item (A) : CASCI space minus the reference space.
\item (X) : External space constructed from determinants obtained by applying excitation operators to the determinants within the CASCI space.
\end{itemize}
These three spaces are orthogonal to one another.\\

\noindent Use them to write the eigenvalue problem:
\begin{equation}
\mathbf{Hc} = \mathbf{cE}
\end{equation}

\begin{equation*}
\begin{bmatrix}
[H^{(RR)}]&[H^{(RA)}]&[H^{(RX)}]\\
[H^{(AR)}]&[H^{(AA)}]&[H^{(AX)}]\\
[H^{(XR)}]&[H^{(XA)}]&[H^{(XX)}]
\end{bmatrix}
\begin{bmatrix}
[c^{(RR)}]&[c^{(RA)}]&[c^{(RX)}]\\
[c^{(AR)}]&[c^{(AA)}]&[c^{(AX)}]\\
[c^{(XR)}]&[c^{(XA)}]&[c^{(XX)}]
\end{bmatrix}
\end{equation*}
\begin{equation}
=
\begin{bmatrix}
[c^{(RR)}]&[c^{(RA)}]&[c^{(RX)}]\\
[c^{(AR)}]&[c^{(AA)}]&[c^{(AX)}]\\
[c^{(XR)}]&[c^{(XA)}]&[c^{(XX)}]
\end{bmatrix}
\begin{bmatrix}
[E^{RR}]& [0]& [0]\\
[0]& [E^{AA}]& [0]\\
[0]& [0]&      [E^{CC}]
\end{bmatrix}
\end{equation}
\noindent From which we can get the set of equations :
\begin{equation}
H^{(RR)}c^{(RR)}+H^{(RA)}c^{(AR)}+H^{(RX)}c^{(XR)}= c^{(RR)}E^{(RR)}
\end{equation}
\begin{equation}
H^{(AR)}c^{(RA)}+H^{(AA)}c^{(AA)}+H^{(AX)}c^{(XA)}= c^{(AA)}E^{(AA)}
\end{equation}
\begin{equation}
H^{(XR)}c^{(RX)}+H^{(XA)}c^{(AX)}+H^{(XX)}c^{(XX)}= c^{(XX)}E^{(XX)}
\end{equation}
\noindent For a particular state, $\nu$, this can be written in index notation as
%\begin{equation}
%\sum_{k} H_{ik}c_{kj}
%+\sum_{f}H_{if}c_{fj}
%+\sum_{x}H_{iz}c_{zj}
%= \sum_{k}c_{ik}E_{\nu},
%\end{equation}
%\begin{equation}
%\sum_{k}  H_{ik}c_{ke}
%+\sum_{f} H_{df}c_{fe}
%+\sum_{x} H_{iz}c_{ze}
%= \sum_{k}c_{df}E_{\nu},
%\end{equation}
%\begin{equation}
%\sum_{k} H_{wk}c_{kx}
%+\sum_{f}H_{wf}c_{fx}
%+\sum_{x}H_{wz}c_{zx}
%= \sum_{z}c_{wz}E_{\nu}.
%\end{equation}
\begin{equation}
\sum_{k} H_{ik}c_{k\nu}
+\sum_{f}H_{if}c_{f\nu}
+\sum_{x}H_{iz}c_{z\nu}
= \sum_{k}c_{ik}E_{\nu},
\end{equation}
\begin{equation}
\sum_{k}  H_{ik}c_{k\nu}
+\sum_{f} H_{df}c_{f\nu}
+\sum_{x} H_{iz}c_{z\nu}
= \sum_{f}c_{f\nu}E_{\nu},
\end{equation}
\begin{equation}
\sum_{k} H_{wk}c_{k\nu}
+\sum_{f}H_{wf}c_{f\nu}
+\sum_{x}H_{wz}c_{z\nu}
= \sum_{z}c_{z\nu}E_{\nu}.
\end{equation}

\noindent In the above and from now on the following convention is used for indexes;
 $i,j,k,l \in R$,  $d,e,f \in A$, and $w,x,y,z \in X$. \\

\noindent  If the reference space, \emph{R}, is well seperated energetically from the 
remainder of the active space then we can assume that $[c^{(RA)}]\approx[c^{(AR)}]\approx 0 $.
This leaves us with the following equations for a state $\nu \in R$:
\begin{equation}
\sum_{k} H_{ik}c_{k\nu}
+\sum_{z}H_{iz}c_{z\nu}
= \sum_{k}c_{\nu k}E_{\nu}
\label{eqn:RrowHc}
\end{equation}
\begin{equation}
\sum_{k} H_{wk}c_{k\nu}
+\sum_{x}H_{wz}c_{z\nu}
= \sum_{z}c_{\nu z}E_{z\nu}.
\label{eqn:XrowHc}
\end{equation}
\noindent Rearranging (\ref{eqn:XrowHc}) to get an expression for $c^{(XR)}_{ y\nu}$ ( $y \in X$, $\nu \in R$);
\begin{equation}
c_{y\nu} = -\sum_{zk}[[H^{(XX)}]-E_{\nu}[I^{(XX)}]]^{-1}_{yz}H_{zk}c_{k\nu},
\label{eqn:cXRfrominverse}
\end{equation}
\noindent where $[I^{(XX)}]$ is the representation of the identity on the external space $X$. This
expression for $ c_{y\nu}$  may be substitued back into (\ref{eqn:RrowHc}) to obtain
\begin{equation}
\sum_{k} H_{ik}c_{k\nu}
+\sum_{y}H_{iy}\sum_{zk}([H^{(XX)}]-E_{\nu}[I^{(XX)}])^{-1}_{yz}H_{zk}c_{k\nu}
= \sum_{k}c_{\nu k}E_{\nu}
\label{eqn:RrowHcsub}
\end{equation}
\noindent This would need to be solved iteratively, due to appearance of $E_{\nu}$ on both
sides of the equation. This is problematic as calculating matrix inverses can prove expensive,
and we would need to calculate the inverse again for every new guess at $E_{\nu}$. To try and deal with this
we write
\begin{equation}
E_{\nu} = E_{\nu}^{0}+\Delta E_{\nu} 
\end{equation}
\noindent leading to
\begin{equation} 
[[H^{(XX)}]-E_{\nu}[I^{(XX)}]]^{-1} = ([C]+[D])^{-1}, 
\end{equation}
\noindent where
\begin{equation}
[C] = [H^{(XX)}]-E^{0}_{\nu}[I^{(XX)}]
\end{equation}
\begin{equation}
[D] = \Delta E_{\nu} [I^{(XX)}] 
\end{equation}  
Now using the identity $(C+D)^{-1} = (1+C^{-1}D)D^{-1}$ we write : 
\begin{equation}
([H^{(XX)}]-E_{\nu}[I^{(XX)}])^{-1} =
(1-\Delta E_{\nu}V^{-1})V , 
\end{equation}
\noindent where $V = (H^{(XX)}-E^{0}_{\nu}[I^{(XX)}])$. Provided
 $\Delta E_{\nu}$ is small we can write this as a series;
\begin{equation}
(1-\Delta E_{\nu}V^{-1})= \sum_{q}^{\infty} (\Delta E_{\nu}V^{-1})^{q}
\end{equation}
Substituing back into (\ref{eqn:RrowHcsub}) and approximating to terms in $(\Delta E ^{2})$ we get
\begin{equation}
\sum_{k} H_{ik}c_{k\nu}
+\sum_{y}H_{iy}\sum_{zkx}(1+ E_{\nu}V^{-1}+ (E_{\nu}V^{-1})^{2})_{yx}V_{xz}H_{zk}c_{k\nu}
= \sum_{k}c_{\nu k}E_{\nu}
\label{eqn:RrowHc_series}
\end{equation}
The advantage of this is that we need only calculate the inverse of $V$, which is independent of $\Delta E$, and so
need only be calculated once during the determination of $c_{k \nu}$.\\

\noindent The members, $|w\rangle $, of space X are given by applying excitation operators to the wavefunctions 
$|\nu\rangle$, which form the reference space; 
\begin{equation}
|w\rangle = a_{s}^{\dagger}a_{t}^{\dagger}a_{p}a_{s}|I\rangle c_{I\nu}
\end{equation}
where I is a determinant in the active space, $s$ and $t$ may be virtual
or active indexes, whilst $p$ and $s$ may be core or active indexes. It is forbidden for
both $s$ and $t$ to be active. Hence the elements of the  $[H^{XX}]$ block are given by
\begin{equation}
\sum_{IJ}
d^{\dagger}_{I\nu}\langle I | a_{s}a_{t}a_{p}^{\dagger}a_{s}^{\dagger}
\hat{H}
a_{s}^{\dagger}a_{t}^{\dagger}a_{p}a_{s}|J\rangle d_{J\nu}
=
d^{\dagger}_{I\nu}\langle I |\hat{E}^{\dagger}_{\Omega} 
\hat{H} \hat{E}^{\dagger}|I\rangle d_{J\nu}.
\end{equation}
It is notable that this means that the definition of the external space is dependent upon the 
reference state $\nu$, which is obviously rather bad...
\end{document}
