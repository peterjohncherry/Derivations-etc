\documentclass[12pt]{article}
\usepackage[utf8x]{inputenc}
\usepackage[english]{babel}
\usepackage[T1]{fontenc}
\usepackage{color}
\usepackage{wrapfig}
\usepackage{amsmath}
\usepackage{amssymb}
\usepackage{textcomp}
\usepackage{array}
\usepackage{booktabs}
\usepackage{subfigure}
\usepackage[font=small,format=plain,labelfont=bf,up,textfont=it,up]{caption}
\usepackage{longtable}
\usepackage{calc}
\usepackage{setspace}
\usepackage{multirow}
\usepackage{hhline}
\usepackage{ifthen}
\usepackage{lscape}
\usepackage{relsize}
\usepackage{bbold}
\usepackage{mathtext}
\usepackage{pdfpages}
\usepackage{geometry}
 \geometry{
 a4paper,
 total={170mm,257mm},
 left=35mm,
 top=25mm,
 bottom=25mm,
 right=20mm,
 }
\usepackage{cite}
\linespread{1.3}

\begin{document}
Multistate Hylleraas type equation for rotated functions $| \tilde{N} \rangle$:
\begin{equation}
E_{L}^{(2)} = \sum_{\Omega\Omega '} \sum_{MN} \langle \tilde{M}  | \hat{T}^{\dagger}_{\Omega ',ML} ( \hat{f} -E^{(f)}_{L} ) \hat{T}_{\Omega,LN} | \tilde{N} \rangle
+ \langle \tilde{M} | \hat{T}^{\dagger}_{\Omega ',ML} \hat{H} | \tilde{N} \rangle.
\end{equation}

Differentiate with respect to T-amplitude, $T^{\dagger}_{\Omega ',ML} $, to get amplitude equation:  
\begin{equation}
\frac{\partial E_{L}^{(2)}}{\partial T^{ML}_{\Omega '}}  =
 \sum_{\Omega} \sum_{N} \langle \tilde{M}  | \hat{E}^{\dagger}_{\Omega '} ( \hat{f} -E^{(f)}_{L} ) \hat{T}_{\Omega,LN} | \tilde{N} \rangle
+ \langle \tilde{M} | \hat{E}^{\dagger}_{\Omega'} \hat{H} | \tilde{N} \rangle = 0
\end{equation}

Left hand side should be stationary with respect to variations in T-amplitudes, in other words, we should minimize the residual $r_{LN,\Omega}$,
\begin{equation}
r_{LN,\Omega}[T_{\Omega,LN}]  =  \langle \tilde{M}  | \hat{E}^{\dagger}_{\Omega '} ( \hat{f} -E^{(f)}_{L} ) \hat{T}_{\Omega,LN} | \tilde{N} \rangle
=0 ,
\end{equation}
with respect to variations in $T_{\Omega, LN}$;
\begin{equation}
r_{LN,\Omega}[T_{\Omega,LN}] = r_{LN,\Omega}[T_{\Omega,LN}+\Delta T_{\Omega,LN}]  
\end{equation}
\begin{equation}
r_{LN,\Omega}[T_{\Omega,LN}] = r_{LN,\Omega}[T_{\Omega,LN}] - \langle \tilde{M}  | \hat{E}^{\dagger}_{\Omega '} ( \hat{f} -E^{(f)}_{L} ) \Delta T_{\Omega,LN} \hat{E}_{\Omega}  | \tilde{N} \rangle 
\end{equation}
leading to  
\begin{equation}
\frac{r_{LN,\Omega}[T_{\Omega,LN}]}{ (r_{LN,\Omega}[T_{\Omega,LN}] - \langle \tilde{M}  | \hat{E}^{\dagger}_{\Omega '} ( \hat{f} -E^{(f)}_{L} ) \hat{E}_{\Omega}  | \tilde{N} \rangle \Delta T_{\Omega,LN})} =1
\end{equation}
assuming we are not too far from convergence
\begin{equation}
r_{LN,\Omega} [T_{\Omega,LN}] \ll  \langle \tilde{M}  | \hat{E}^{\dagger}_{\Omega '} ( \hat{f} -E^{(f)}_{L} ) \hat{E}_{\Omega}  | \tilde{N} \rangle \Delta T_{\Omega,LN}
\end{equation}
so 
\begin{equation}
\Delta T_{\Omega, LN } \approx  \frac{r_{LN,\Omega}[T_{\Omega,LN}]}{ \langle \tilde{M}  | \hat{E}^{\dagger}_{\Omega '} ( \hat{f} -E^{(f)}_{L} ) \hat{E}_{\Omega}  | \tilde{N\rangle}}.
\end{equation}
The states $\{|\tilde{N}\rangle\}$ diagonalize the Fock operator, hence $M = N$. Furthermore, the off diagonal elements of the Fock operator
(in terms of molecular orbital indexes) are small, so we need only consider terms where $\Omega = \Omega '$. This leads to 
\begin{equation}
\Delta T_{\Omega, LN } \approx  \frac{r_{LN,\Omega}[T_{\Omega,LN}]}{ \langle \tilde{N}  | \hat{E}^{\dagger}_{\Omega} ( \hat{f} -E^{(f)}_{L} ) \hat{E}_{\Omega}  | \tilde{N\rangle}}.
\end{equation}
\end{document}
