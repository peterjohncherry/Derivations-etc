\documentclass[12pt]{article}
\usepackage[utf8x]{inputenc}
\usepackage[english]{babel}
\usepackage[T1]{fontenc}
\usepackage{color}
\usepackage{wrapfig}
\usepackage{amsmath}
\usepackage{amssymb}
\usepackage{textcomp}
\usepackage{array}
\usepackage{booktabs}
\usepackage{subfigure}
\usepackage[font=small,format=plain,labelfont=bf,up,textfont=it,up]{caption}
\usepackage{longtable}
\usepackage{calc}
\usepackage{setspace}
\usepackage{multirow}
\usepackage{hhline}
\usepackage{ifthen}
\usepackage{lscape}
\usepackage{relsize}
\usepackage{bbold}
\usepackage{mathtext}
\usepackage{pdfpages}
\usepackage{geometry}
 \geometry{
 a4paper,
 total={170mm,257mm},
 left=35mm,
 top=25mm,
 bottom=25mm,
 right=20mm,
 }
\usepackage{cite}
\linespread{1.3}

\begin{document}
Using the anti-commutation relation $\{a_{i}, a_{j}\} = 0$, we can write 
\begin{equation}
a^{\dagger}_{i'}  a^{\dagger}_{j'}a^{\dagger}_{k'} =  \epsilon_{i'j'k'} 
a^{\dagger}_{i}  a^{\dagger}_{j}a^{\dagger}_{k}    
\end{equation}
where $\{i',j',k\} = P \{i',j',k\}$ , $P$ is some permutation operator, and $\epsilon_{i'j'k'}$ is 
the Levi-Cevita symbol.\\

\noindent This can then be applied to contributions to the rdm derivatives:
\begin{equation}
\langle I | a^{\dagger}_{i'}  a^{\dagger}_{j'}a^{\dagger}_{k'}  a_{l'}  a_{m'}a_{n'}| J \rangle c_{J}  =  
\epsilon_{i'j'k'}
\epsilon_{l'm'n'}
\langle I | a^{\dagger}_{i}  a^{\dagger}_{j}a^{\dagger}_{k}  a_{l}  a_{m}a_{n}| J \rangle c_{J}   
\end{equation}
Each of the $\epsilon_{xyz}$ corresponds exclusively to either creation or annihilation operators, hence
the reorderings with which they are associated correspond only to alterations of the phase, and do not affect
the total projection. \\

\noindent Recall the identity
\begin{equation}
\epsilon_{ijk}\epsilon_{lmn}
=
 \delta_{il}\delta_{jm}\delta_{kn}
-\delta_{im}\delta_{jl}\delta_{kn}
+\delta_{im}\delta_{jn}\delta_{kl}
-\delta_{in}\delta_{jm}\delta_{kl}
+\delta_{in}\delta_{jl}\delta_{km}
-\delta_{il}\delta_{jn}\delta_{km}
= \delta_{ijk}^{lmn}
\end{equation}
Where is the Kronecker delta. Each of the Dirac delta functions corresponds
to a contraction between a creation and annihilation operator.
This can be used to rewrite the contraction between the rdm derivative and the "A"-tensor:

\begin{equation}
\sum_{\substack{ ijk \\ lmn}}\Gamma^{I}_{ijklmn}A_{ijklmn}  =  
\end{equation}
\begin{equation}
\sum_{\substack{ ijk \\ lmn}} \sum_{J}
 \langle I | a^{\dagger}_{i}  a^{\dagger}_{j}a^{\dagger}_{k}  a_{l}  a_{m}a_{n}| J \rangle c_{J} A_{ijklmn}  =  
\end{equation}
\begin{equation}
\sum_{\substack{ ijk \\ lmn \\ i<j<k \\  l<m<n }}
\sum_{\substack{ \{i'j'k'\} \\ \{l'm'n'\} }}
\sum_{J}
\epsilon_{i'j'k'}
\epsilon_{l'm'n'}
\langle I | a^{\dagger}_{i}  a^{\dagger}_{j}a^{\dagger}_{k}  a_{l}  a_{m}a_{n}| J \rangle c_{J}   A_{ijklmn}
\end{equation}
\begin{equation}
\sum_{\substack{ ijk \\ lmn \\ i<j<k \\  l<m<n }}
\sum_{\substack{ \{i'j'k'\} \\ \{l'm'n'\} }}
\sum_{J}
\delta_{i'j'k'}^{l'm'n'}   
\langle I | a^{\dagger}_{i}  a^{\dagger}_{j}a^{\dagger}_{k}  a_{l}  a_{m}a_{n}| J \rangle c_{J}   A_{ijklmn}
\end{equation}
\begin{equation*}
\sum_{\substack{ ijk \\ lmn \\ i<j<k \\  l<m<n }}
\sum_{\substack{ \{i'j'k'\} \\ \{l'm'n'\} }}
\sum_{J}
(\delta_{il}\delta_{jm}\delta_{kn}
\langle I | a^{\dagger}_{i}  a^{\dagger}_{j}a^{\dagger}_{k}  a_{l}  a_{m}a_{n}| J \rangle c_{J}   A_{ijklmn}
\end{equation*}
\begin{equation*}
-\delta_{im}\delta_{jl}\delta_{kn}
\langle I | a^{\dagger}_{i}  a^{\dagger}_{j}a^{\dagger}_{k}  a_{l}  a_{m}a_{n}| J \rangle c_{J}   A_{ijklmn}
\end{equation*}
\begin{equation*}
+\delta_{im}\delta_{jn}\delta_{kl}
\langle I | a^{\dagger}_{i}  a^{\dagger}_{j}a^{\dagger}_{k}  a_{l}  a_{m}a_{n}| J \rangle c_{J}   A_{ijklmn}
\end{equation*}
\begin{equation*}
-\delta_{in}\delta_{jm}\delta_{kl}
\langle I | a^{\dagger}_{i}  a^{\dagger}_{j}a^{\dagger}_{k}  a_{l}  a_{m}a_{n}| J \rangle c_{J}   A_{ijklmn}
\end{equation*}
\begin{equation*}
+\delta_{in}\delta_{jl}\delta_{km}
\langle I | a^{\dagger}_{i}  a^{\dagger}_{j}a^{\dagger}_{k}  a_{l}  a_{m}a_{n}| J \rangle c_{J}   A_{ijklmn}
\end{equation*}
\begin{equation*}
-\delta_{il}\delta_{jn}\delta_{km}
\langle I | a^{\dagger}_{i}  a^{\dagger}_{j}a^{\dagger}_{k}  a_{l}  a_{m}a_{n}| J \rangle c_{J}   A_{ijklmn}
\end{equation*}
\begin{equation*}
=\sum_{\substack{ ijk \\ lmn \\ i<j<k \\  l<m<n }}
\sum_{\substack{ \{i'j'k'\} \\ \{l'm'n'\} }}
\sum_{J}
\langle I | a^{\dagger}_{l}  a^{\dagger}_{m}a^{\dagger}_{n}  a_{l}  a_{m}a_{n}| J \rangle c_{J}   A_{lmnlmn}
\end{equation*}
\begin{equation*}
\langle I | a^{\dagger}_{m}  a^{\dagger}_{l}a^{\dagger}_{n}  a_{l}  a_{m}a_{n}| J \rangle c_{J}   A_{mlnlmn}
\end{equation*}
\begin{equation*}
\langle I | a^{\dagger}_{m}  a^{\dagger}_{n}a^{\dagger}_{l}  a_{l}  a_{m}a_{n}| J \rangle c_{J}   A_{mnllmn}
\end{equation*}
\begin{equation*}
\langle I | a^{\dagger}_{n}  a^{\dagger}_{m}a^{\dagger}_{l}  a_{l}  a_{m}a_{n}| J \rangle c_{J}   A_{nmllmn}
\end{equation*}
\begin{equation*}
\langle I | a^{\dagger}_{n}  a^{\dagger}_{l}a^{\dagger}_{m}  a_{l}  a_{m}a_{n}| J \rangle c_{J}   A_{nlmlmn}
\end{equation*}
\begin{equation*}
\langle I | a^{\dagger}_{l}  a^{\dagger}_{n}a^{\dagger}_{m}  a_{l}  a_{m}a_{n}| J \rangle c_{J}   A_{lnmlmn}
\end{equation*}

\begin{equation*}
=\sum_{\substack{ ijk \\ lmn \\ i<j<k \\  l<m<n }}
\sum_{\substack{ \{i'j'k'\} \\ \{l'm'n'\} }}
\sum_{J}
\langle I | a^{\dagger}_{l}a_{l} a^{\dagger}_{m}a_{m}a^{\dagger}_{n}a_{n}| J \rangle c_{J} 
A_{lmnlmn} +A_{mlnlmn} +A_{mnllmn} +A_{nmllmn} +A_{nlmlmn} +A_{lnmlmn}
\end{equation*}














\noindent Hence the rdm derivative does not need to be calculated; it's anti-symmetry under
permutation of the creation/annihilation indexes means it is possible to rewrite it
 entirely in terms of delta functions.\\

\noindent It looks as though matrix elements between different determinants vanish, which seems wrong. However,
these interactions are still effectively present; elements of $A_{ijklmn}$ corresponding to orbitals not in $|J\rangle$ will 
be weighted by $c_{J}$ due to the contractions and nature of the summation.

\noindent This will apply to all normal ordered operators:
\begin{equation}
\sum_{\substack{ i_{1}...i_{N} \\ j_{1}...j_{N} \\ i_{1}<i_{2}...<i_{N} \\ j_{1}<j_{2}...<j_{N} }}
\sum_{\substack{ \{i'_{1}...i'_{N}\} \\ \{j'_{1}...j'_{N}\} }}
\sum_{J}
\epsilon_{i_{1}....i_{N}}
\epsilon_{j_{1}....j_{N}}
\langle I | a^{\dagger}_{i_{1}}...a^{\dagger}_{i_{N}} a_{j_{1}}...a_{j_{n}}| J \rangle c_{J}   A_{i_{1}...i_{N}j_{1}...j_{N}}
\end{equation}





\end{document}
