\documentclass[12pt]{article}
\usepackage[utf8x]{inputenc}
\usepackage[english]{babel}
\usepackage[T1]{fontenc}
\usepackage{color}
\usepackage{wrapfig}
\usepackage{amsmath}
\usepackage{amssymb}
\usepackage{textcomp}
\usepackage{array}
\usepackage{booktabs}
\usepackage{subfigure}
\usepackage[font=small,format=plain,labelfont=bf,up,textfont=it,up]{caption}
\usepackage{longtable}
\usepackage{calc}
\usepackage{setspace}
\usepackage{multirow}
\usepackage{hhline}
\usepackage{ifthen}
\usepackage{lscape}
\usepackage{relsize}
\usepackage{bbold}
\usepackage{mathtext}
\usepackage{pdfpages}
\usepackage{geometry}
 \geometry{
 a4paper,
 total={170mm,257mm},
 left=35mm,
 top=25mm,
 bottom=25mm,
 right=20mm,
 }
\usepackage{cite}
\linespread{1.3}

\begin{document}

need program

current example is TCE.

generated code, very effective.

Not geared towards four component calculations, yet. Similarly, primarily designed for coupled cluster, 
and not yet adapted for cases where the wavefunction is constructed using multiple active spaces.

Furthermore, would like to generate task lists for evaluating expressions "on the fly". Flexibility will
likely come with speed drawbacks (unless the user interface were made very complicated), but hopefully 
will enable quicker testing of theories, even by those averse to code development.

Though the software design was heavily informed by currently existing methods, ... list , 
the emphasis on these capabilities, the handling of algebraic expressions, as well as the overall
program structure differs substantially from that of other similar programs. 

This method has been used to calculate quantities such as nuclear gradients [ref] in the 4-component XMS-CASPT2 framework,
and further exploration of other quantities,  specifically NMR EPR parameters/Jahn-Teller effects is planned.






Instead interface so done on the fly.

No substantial advantage

\end{document}
