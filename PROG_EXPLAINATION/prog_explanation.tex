\documentclass[12pt]{article}
\usepackage[utf8x]{inputenc}
\usepackage[english]{babel}
\usepackage[T1]{fontenc}
\usepackage{color}
\usepackage{wrapfig}
\usepackage{amsmath}
\usepackage{amssymb}
\usepackage{textcomp}
\usepackage{array}
\usepackage{booktabs}
\usepackage{subfigure}
\usepackage[font=small,format=plain,labelfont=bf,up,textfont=it,up]{caption}
\usepackage{longtable}
\usepackage{calc}
\usepackage{setspace}
\usepackage{multirow}
\usepackage{hhline}
\usepackage{ifthen}
\usepackage{lscape}
\usepackage{relsize}
\usepackage{bbold}
\usepackage{mathtext}
\usepackage{pdfpages}
\usepackage{geometry}
 \geometry{
 a4paper,
 total={170mm,257mm},
 left=35mm,
 top=25mm,
 bottom=25mm,
 right=20mm,
 }
\usepackage{cite}
\linespread{1.3}

\begin{document}

The program is intended to help with evaluation of the many index expressions which arise in multi-reference pertrbation theory. 
Essentially, it aims to evaluate one of the follwing three kinds of expressions, the simplest of which is
\begin{equation}
\langle M | \hat{X} \hat{Y} ... | N \rangle,
\end{equation}
where  $| N \rangle$  is a multireference wavefunction represented as a linear combination of determinants, $|I\rangle $;
\begin{equation}
|N\rangle = \sum_{I} c_{I}^{N}| I \rangle.
\end{equation} 
It can also calculate the derivative of the above expression with respect to one of the coefficients $c_{I}$:
\begin{equation}
\langle M | \hat{X} \hat{Y} ... | I \rangle.
\end{equation}
Similary, it can calculate the derivative with respect to   

In the second quantized representation we can write this as
\begin{equation}
z = \sum_{\substack{ x_{1}x_{2}...\\ y_{1}y_{2}... \\ ...}} X_{x_{1}x_{2}...} Y_{y_{1}y_{2}...} ...
\sum_{I}\sum_{J}
\langle I | a^{\dagger}_{x_{1}} a_{x_{2}}...a^{\dagger}_{y_{1}}a_{y_{2}}....| J \rangle 
 c^{\dagger}_{I}c_{J}
\end{equation}

\begin{equation}
x = \rangle M | \hat{A} \hat{B}...... | N \rangle
\end{equation}


where $x$ is a scalar. This is  


\end{document}
