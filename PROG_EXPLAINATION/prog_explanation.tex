\documentclass[12pt]{article}
\usepackage[utf8x]{inputenc}
\usepackage[english]{babel}
\usepackage[T1]{fontenc}
\usepackage{color}
\usepackage{wrapfig}
\usepackage{amsmath}
\usepackage{amssymb}
\usepackage{textcomp}
\usepackage{array}
\usepackage{booktabs}
\usepackage{subfigure}
\usepackage[font=small,format=plain,labelfont=bf,up,textfont=it,up]{caption}
\usepackage{longtable}
\usepackage{calc}
\usepackage{setspace}
\usepackage{multirow}
\usepackage{hhline}
\usepackage{ifthen}
\usepackage{lscape}
\usepackage{relsize}
\usepackage{bbold}
\usepackage{mathtext}
\usepackage{pdfpages}
\usepackage{geometry}
 \geometry{
 a4paper,
 total={170mm,257mm},
 left=35mm,
 top=25mm,
 bottom=25mm,
 right=20mm,
 }
\usepackage{cite}
\linespread{1.3}

\begin{document}

\section{background}

In experiments we measure "observable values". Denote one such variable as $x$.
To be clear $x$ is a number (not a matrix, operator, function  etc.). Each
observable corresponds to an operator $\hat{X}$. The hat $\hat{ }$ means this is
an operator\footnote{Instantiations of the TensOp class contain information
about the operators}.\\

Experimental systems are said to be in different "states". We try to describe
state $M$ with a "wavefunction" $\Psi_{M}$ . Note this wavefuction describes
the behavious of all the particles in the system (electrons in this case).  The
basic properties of these states are defined in the StateInfo object, and also
in CASPT2\_ALT. \\

The wavefunction is described as a linear combination of other wavefunctions:
\begin{equation}
\Psi_{M} = \sum_{I} c_{I}\Phi_{I},
\label {eqn:CI_wfn}
\end{equation}

Each $\Phi$ corresponds to a different configuration of N particles, and the
are $c_{I}$ are coefficients defining how much each configuration contributes to the  total
wavefunction,$Psi$\footnote{The array with elements $c_{I}$ is the CIVec
object}. It is very important to note that the total wavefunction $\Psi$
\emph{cannot} usually be described by any single configuration of particles. \\ 

To get the value of observable $x_{M}$ associated with state $\Psi_{M}$ we evaluate.

\begin{equation}
\langle \Psi_{M} | \hat{X}| \Psi_{M} \rangle
\end{equation}

The $\langle |$ part is the "Bra" member of the GammaInfo object, whilst the "Ket" is the $| \rangle$.
\footnote{These obtained results are stored in the scalar\_results\_map}.\\

The $\langle | | \rangle$ is associated with the method used to evaluate these
terms.  Do not worry about it, just bear in mind that $\hat{X} |\Psi_{M}
\rangle$ is \emph{not} $\hat{X}$ times $\Psi_{M}$; neither $\hat{X}$ nor $\Psi$ have any
defined numerical representation. The idea is that provided we represent $\hat{X}$
and $\Psi$ in a manner which preserves their logical structure and relationship
to one another, the result of the above expression is the same. \\
 
This has informed how the program is written; in general all classes with info
in their name contain \emph{no} actual data about the representation of the wavefunction or operators,
they only contain information about their structure, and instructions how to best evaluate 
terms which involve them. Determining these instructions is what is done in the functions in the
gamma\_generator, TensOp, MultiTensOp, CtrTensOp and CtrMultiTensOp classes, whilst all the actual
work is carried out in the more ???\_computer and ????\_arithmetic classes.\\

To understand how each of these classes relate to one another we need to discuss the physics a bit more.\\

Most observables can be boiled down to interactions between pairs of electrons. This means we can rewrite the above expression as

\begin{equation}
\sum_{\substack{electron \\ pairs}} \langle \Psi_{M} | \hat{X}^{2el}| \Psi_{M} \rangle
\label{eqn:2el_inter}
\end{equation}

Where $\hat{X}^{2el}$ is a two electron operator, i.e., we calculate the
contributions to the observable from each pair of electrons seperately and then
sum them up. In the code I refer to these are BraKets, and information about them
is stored in the BraKet class.\\ 

However, as mentioned above, the wavefunction $\Psi_{M}$ does not
correspond to single cofiguration of particles, so it is not clear how the
summation over electron pairs can be accomplished. In order to do so we must
use the representation defined in equation (\ref{eqn:CI_wfn}). Substituting
equation (\ref{eqn:CI_wfn}) into (\ref{eqn:2el_inter}) we obtain:

\begin{equation}
\sum_{\substack{electron \\ pairs}}\sum_{IJ}
 c^{M}_{I}\langle \Phi_{I} | \hat{X}^{2el} | \Phi_{J} \rangle c_{J}^{M}
\label{eqn:2el_inter_2}
\end{equation}

Now, instead of calculating a single interaction between two wavefunctions with ill-defined particles,
we calculate a lot of interactions between particles with well defined particles. This can now be rewritten: 

\begin{equation}
\sum_{ijkl}\sum_{IJ} c_{M}^{I}\langle \Phi_{I} | \hat{a}_{i}^{\dagger}\hat{a}_{j}^{\dagger}\hat{a}_{k}\hat{a}_{l} | \Phi_{J} \rangle c_{J}^{M} X_{ijkl}
\label{eqn:2el_expectation}
\end{equation}

This is a bit nasty looking, but it is the basic kind of expression I am evaluating, so it is worth going through properly:
\begin{itemize}
\item Here $\hat{a}_{l}$ "annihlates" particle $l$ in $\Phi_{M}$ whilst
$\hat{a}^{\dagger}_{i}$ "creates" particle $i$. So we annihilate two particles in $\Phi_{J}$
and then create two in $\Phi_{I}$. In other words we are exciting from I to J.
These $\hat{a}$ operators are referred to as aops in the code. They have no numerical repsentation,
but describe how configurations relate to one another.
\item The $X_{ijkl}$ are elements of a tensor (here, a 4-index array of floats) which describes how
the operator $\hat{X}$ influences the interaction of particles $i$, $j$, $k$ and $l$.
\footnote{These tensors are stored in TensOp\_data\_map, note that the corresponding in TensOp object
does not contain this data, it only has a name so it can be fetched from the map. }  \\
\item The matrix with elements
$\sum_{IJ} c^{M}_{I}\langle \Phi_{I} | \hat{a}_{i}^{\dagger}\hat{a}_{j}^{\dagger}\hat{a}_{k}\hat{a}_{l} | \Phi_{J} \rangle c_{J}^{M}  $
is referred to as the (2-electron reduced) density matrix, and defines how the different configurations interact. For various reasons,
I will refer to them as gammas, here and in the code. \footnote{The elements of gammas are stored in Gamma\_data\_map, are
calculated in the gamma\_computer class. Which gammas we need is for a given expressions is
determined in the gammagenerator class, and information about the a given gamma is defined in a gamma\_info object.}.\\
\end{itemize}
Writing expressions like this has several advantages due to properties of the configurations
 $Phi_{J}$ and the operators $\hat{a}$. By definition
\begin{equation}
\langle \Phi_{I} | \Phi_{J} = 
\begin{cases}
1 \text{\  \ \ \  \ \ \ \ if } \Phi_{I} | \Phi_{J}\\ 
0 \text{ \ \ otherwise }.
\end{cases}
\label{eqn:overlap}
\end{equation}

In otherwords, if $\Phi_{I} | \Phi_{J}$ aren't the same, the term vanishes. However, in equantion 
\label{eqn:2el_expectation} there are several particle creation and destruction operators between the 
two $\Phi_{I} | \Phi_{J}$, and these operators act to transform configurations into one another. For example suppose
I have a configuration $\Phi(1,2,3)$, which contains particles 1,2, and 3, and another configuration $\Phi(1,2,4)$ 
containing particles 1,2 and 4. We can use the creation and annhiliation operators, $\hat{a}_{4}$
and $\hat{a}_{3}^{\dagger}$  to transform between these two configurations;
\begin{equation}
|\Phi(1,2,3)\rangle = \hat{a}_{3}^{\dagger}\hat{a}_{4} |\Phi(1,2,4)\rangle.
\end{equation}
What's happening here is that the operator $\hat{a}_{4}$ destroys particle $4$ in $\Phi(1,2,4)$, and then 
operator $\hat{a}^{\dagger}_{3}$ fills this whole with particle $3$. So if we now revist equation (\ref{eqn:overlap})
we can now get the relation 
\begin{equation}
\langle \Phi(1,2,4) |\hat{a}_{i}^{\dagger}\hat{a}_{j} \Phi(1,2,4) =
\begin{cases}
1 \text{\  \ \ \  \ \ \ \ if } i=3 \text{ \ \ and \ \ } j =4 \\
0 \text{ \ \ otherwise }. 
\end{cases}
\label{eqn:overlap_specific}
\end{equation}

As we can see now, which terms contribute to \label{eqn:2el_expectation} depend on which configurations $\Phi$ we are using to 
define the total wavefunciton $\Phi$, and which particles our operator $\hat{X}$ acts on. For example, we may know that 
our observable  is only significanlty influenced by those electrons in the outermost shells, enabling us to skip many terms in the 
summaiton. Furthermore, our operator $\hat{X}$ may have internal symmetries, reducing the number of unique elements $X_{ijkl}$ 
we need to store.\\

The purpose of my code is to take advantage of these symmetries and range constraints. It should do it automatically,
and for any number of operators, whose symmetries and range constraints are specified by functions input by the used. The 4-index term above
is not actually problematic. But to calculate the things I am interested in I have a dozen or so terms looking like this
\begin{equation}
\sum_{IJ} c_{NI}\langle \Phi_{N} |
\hat{a}_{i}
\hat{a}_{j}
\hat{a}_{k}^{\dagger}
\hat{a}_{l}^{\dagger}
\hat{a}_{m}^{\dagger}
\hat{a}_{n}
\hat{a}_{o}^{\dagger}
\hat{a}_{p}^{\dagger}
\hat{a}_{q}
\hat{a}_{s}
 | \Phi_{J} \rangle c_{JM}  
X^{*}_{klij}Y_{mn}Z_{opqs}
\label{eqn:original}
\end{equation}
We can never store the relevant arrays; they are too large. However, the creation and annhilation operators possess a
permutation symmetry which enables us to rewrite
the above 10-index term, as several (often ~100 or so) terms with fewer indexes:
=
\begin{equation}
\sum_{\nu}^{ \nu \in \{\Omega^{(1)}_{\nu}\}}
\sum_{ijklmnopqr}
\gamma_{k'l'm'n'o'p'q'r'}
X^{*}_{klij}Y_{mn}Z_{opqs}
\delta_{i'j'}
\end{equation}

\begin{equation}+
\sum_{\nu}^{ \nu \in \{\Omega^{(2)}_{\nu}\}}
\sum_{ijklmnopqr}
\gamma_{m'n'o'p'q'r'}
X^{*}_{klij}Y_{mn}Z_{opqs}
\delta_{i'j'}\delta_{k'l'}
\end{equation}

\begin{equation}+
\sum_{\nu}^{ \nu \in \{\Omega^{(3)}_{\nu}\}}
\sum_{ijklmnopqr}
\gamma_{o'p'q'r'}
X^{*}_{klij}Y_{mn}Z_{opqs}
\delta_{i'j'}\delta_{k'l'}\delta_{m'n'}
\end{equation}
\begin{equation}+
\sum_{\nu}^{ \nu \in \{\Omega^{(4)}_{\nu}\}}
\sum_{ijklmnopqr}
\gamma_{q'r'}
X^{*}_{klij}Y_{mn}Z_{opqs}
\delta_{i'j'}\delta_{k'l'}\delta_{m'n'}\delta_{o'p'}
\end{equation}
\begin{equation}
\sum_{\nu}^{ \nu \in \{\Omega^{(5)}_{\nu}\}}
\sum_{ijklmnopqr}
X^{*}_{klij}Y_{mn}Z_{opqs}
\delta_{i'j'}\delta_{k'l'}\delta_{m'n'}\delta_{o'p'}\delta_{q'r'}
\label{eqn:wicked}
\end{equation}

where 
\begin{equation}
\gamma_{ijkl} =\sum_{IJ} c^{M}_{I}\langle \Phi_{I} | \hat{a}_{i}^{\dagger}\hat{a}_{j}^{\dagger}\hat{a}_{k}\hat{a}_{l} | \Phi_{J} \rangle c_{J}^{M}  
\end{equation}

The primed indices are obtained by applying the permutation operation
$\Omega$ to the set of unprimed indexes, i.e.,
\begin{equation}
 \{i'j'k'l'm'n'o'p'q'r'\} = \Omega\{ijklmnopqr\}.
\end{equation}
The outer summation in equation \ref{eqn:wicked} runs over different permutations (there is a way of figuring out
which ones you need). So this outer summation is a sum over possible relabellings of the indexes,
not values of the indexes themselves.\\

The $\delta_{x'y'}=  0$ if $x'\neq y'$ and 1 otherwise.  Hence, the permutation in the outer summation
combined with the delta functions is expresses that each different term in the summation involves
the same baisc tensors, but with different indexes set equal.
For example, in the four index term;
\begin{equation}
\sum_{\nu}^{ \nu \in \{\Omega^{(5)}_{\nu}\}}
\sum_{ijklmnopqr}
\gamma_{o'p'q'r'}
X^{*}_{klij}Y_{mn}Z_{opqs}\delta_{i',j'}\delta_{k'l'}\delta_{m'n'}
\end{equation}
one contribution might have $i'=k$ , $j'=k$, $k'=q$, $l'=s$, $m'=j$ and $n'=m$;
\begin{equation}
\sum_{ijklmnopqr}
\gamma_{ilnr}
X^{*}_{klij}Y_{mn}Z_{opqs}\delta_{ko}\delta_{qs}\delta_{jm}= 
\sum_{ilnr}
\gamma_{ilnr}
\sum_{jkmoqs}
X^{*}_{klij}Y_{mn}Z_{opqs}\delta_{ko}\delta_{qs}\delta_{jm}= 
\end{equation}

So in equation (\ref{eqn:wicked}) some of the indexes of X, Y and Z are being
forced to be the same, which reduces the dimensions of everything involved and
splits things up into two smaller operations.
Setting two indexes the same like this is referred to as a contraction. In the code
I often refer to pair of numbers $\{i,j\}$ as a contraction, a ctr or a delta.\\

What my code does is reorders the code to reduce the number of distinct combinations of contractions,
whilst also trying to reduce the dimension of the problem as much as possible. It then
generates an list of operations which need to be performed to evaluate these
expressions; the difficulty is that there are numerous different orders in
which these expressions can be evaluated, some of which are much more efficient
than others. So basically the code contains an automatic equation rearranger. 
This is necessary, as the number of terms in the rearranged expression is large,
so it cannot be done by hand for all different quantities.\\

How to best rearrange the terms is dependent on the range constraints on the
indexes and the internal symmetries of the operators. Getting it to take 
advantage of this is the hard part. Furthermore, there  are often lots
of different BraKets being calculated. The contributions to each
BraKet are different but ultimately many can be merged together, and some 
can be known to be equivalent or vanish due to properties of the system. Accounting
for this can be tricky. \\

The meat of all this is rearrangement is done in CtrTensOp and
gamma\_generator.  Information about a block of tensor is stored in a CtrTensOp
object (CTP = contracted tensor part).  For example, $X_{ijkl}$ are the elements
of a CTP.  If I have multiple tensors in one expression ($XYZ$), I stored this
information in CtrMultiTensOp object (CMTP = contracted multi tensor part).
For example, $T_{qrst}X_{ijkl}$ are elements of a CTP.  Contracting indexes on
two different tensors takes us from a CMTP to a CTP, e.g.,
$\sum_{t}T_{qrst}X_{ijkl}\delta_{ti}=A_{qrsjkl} $.  The effect of contracting
indexes on different tensors effectively fuses those tensors so they can no
longer be seperated, and their symmetries and contraints must be combined.
Generally speaking I start with a CMTP object, and then must contract it to get
a CTP object, and then contract this in lots of different ways. This is done in
a full\_contract in CtrTensOp.  \\

It is important to note that the tensors are split up into blocks corresponding
to different ranges (these range blocks are nothing to do with the
parallelism). Each CTP and CMTP object only corresponds to a certain range
block. So there are several different CMTP objects for every tensor.  A list of
all possible ranges for a given tensor is specfied in the all\_ranges member of
the TensOp and MultiTensOp classes.\\

Which terms are includes in equation\ref{eqn:wicked} are dependent on the range
blocks. The list of contributing terms, with their range blocks, is determined
in gamma generator.  Which terms need to be calculated (there may be
intermediate terms, some terms can be combined etc.,) is determined in the full
contract functions in CtrTensOp. \\

CtrTensOp generate a task list, which is executed by expression computer.
Expression computer then calls tensor arithmetic,  and gamma computer, The
former contains the acual linear algebra routines for contracting the tensors,
whilst the latter contains routines for calculating 
the $\gamma_{ijkl} =\sum_{IJ} c_{NI}\langle \Phi_{N} |
\hat{a}_{i}^{\dagger}
\hat{a}_{j}
\hat{a}_{k}^{\dagger}
\hat{a}_{l}
| \Phi_{J} \rangle c_{JM}  $ type terms.

In conclusion, the basic proceedure is 
\begin{itemize}
\item Things start in src/smith/caspt/CASPT2\_ALT.cc, but most of the code is in /src/smith/wicktool .
\item First instantiate a Sys\_Op object which contains loads of background information. 
\item Put information about the wavefunction (states\_info objects) and  operators (TensOp objects) into the Sys\_Op object.
\item TensOp will define all possible sub blocks and contractions, and identify equivalencies, but will not do any actual calculations.
\item Instantiate a Expression\_Info object which contains the actual mathematical expressions I want to calculate.
\item Expression Info will then generate some  BraKet objects, each of which corresponds to a term like . 
\item The Bracket will then run the gamma generator to see which terms contribute.
\item The gamma\_generator will produce the gamma\_info objects, and also CtrTenOp objects containing information about the terms
we need to calculate.
\item Full contract (in CtrTensOp) then runs on each of these CTP and CMTP objects, which will provide a list of the linear algebra operations
we need to do.
\item Each gamma object will have a seperate task list (ACompute\_list), which contains all the contracted operators which need 
to be contracted with that gamma (n.b. matrix multiplication is a special case of contraction). Each of these lists is put into the G\_to\_A\_map;
the key is the name of the gamma matrix, whilst the mapped object is the ACompute list.
\item Then the initial data about the wavefunction and operators to generate an expression\_computer member of Sys\_Op. 
\item This expression\_computer object is then used to execute the various lists as described above.\\
\end{itemize}
\end{document}
