\documentclass[12pt]{article}
\usepackage[utf8x]{inputenc}
\usepackage[english]{babel}
\usepackage[T1]{fontenc}
\usepackage{color}
\usepackage{wrapfig}
\usepackage{amsmath}
\usepackage{amssymb}
\usepackage{textcomp}
\usepackage{array}
\usepackage{booktabs}
\usepackage{subfigure}
\usepackage[font=small,format=plain,labelfont=bf,up,textfont=it,up]{caption}
\usepackage{longtable}
\usepackage{calc}
\usepackage{setspace}
\usepackage{multirow}
\usepackage{hhline}
\usepackage{ifthen}
\usepackage{lscape}
\usepackage{relsize}
\usepackage{bbold}
\usepackage{mathtext}
\usepackage{pdfpages}
\usepackage{geometry}
 \geometry{
 a4paper,
 total={170mm,257mm},
 left=35mm,
 top=25mm,
 bottom=25mm,
 right=20mm,
 }
\usepackage{cite}
\linespread{1.3}

\begin{document}
\section{Summary}
\noindent If the system is composed of a number of different spins, then
we should describe it using a tensor product of 
a number of different subspaces, each of which corresponds to one of these
spins. The 



\noindent Normally, this is not the case for EPR/NMR tensor calculations; 
the basis in which the perturbing operators are represented
is comprised of wavefunctions obtained from relativistic Hamiltonians, 
the eigenfunctions of which do not, usually, satisfy this criterion\footnote{
The alternative is that the  wavefunctions which have spin-orbit effects included via some 
perturbational method, but this will result in the same problem.}.\\

\noindent In the following, a basis composed of non-relativistic wavefunctions
is used to of the relevant operators. It is then shown how the time
dependence of operator expectation values can be expressed as sum of
contributions from a several different spins, and the interactions between them. 

\subsection{Describing the system as several interacting spins} 
\noindent  The eigenfunctions of the Dirac-Breit-Gaunt Hamiltonian can be expressed as linear combinations
of eigenfunctions of the non-relativistic Hartree-Fock Hamiltonian:
\begin{equation}
|\Psi_{i}\rangle=\sum_{\gamma}c_{i,\gamma}|\Phi_{\gamma}\rangle
\end{equation}
\noindent the corresponding density operator $ |\Psi_{i}\rangle\langle\Psi_{i}| $ for relativistic wavefunction, $|\Psi_{i}\rangle$,
represented in the basis formed from the non-relativistic wavefunctions is then
\begin{equation}
[\rho_{i}] = 
\begin{bmatrix}
c_{i,1}c^{*}_{i,1}& c_{i,1}c^{*}_{i,2}0 & c_{i,1}c^{*}_{i,3}0 & ...\\
c_{i,2}c^{*}_{i,1}& c_{i,2}c^{*}_{i,2}0 & c_{i,2}c^{*}_{i,3}0 & ...\\
c_{i,3}c^{*}_{i,1}& c_{i,3}c^{*}_{i,2}0 & c_{i,3}c^{*}_{i,3}0 & ...\\
... & ... & ... & ...
\end{bmatrix}
\end{equation}
\noindent The indexes correspond to the indexes of the eigenfunctions in the non-relativistic
Hamiltonian.\\

The density operator, $\hat{\rho}(t)$, associated with the relativistic wavefunction 
can also be represented in the basis formed from the non-relativistic wavefunctions as;
\begin{equation}
\begin{bmatrix}
\rho(0)_{11} &\rho(0)_{12} & \rho(0)_{13} & ...\\
\rho(0)_{21} &\rho(0)_{22} & \rho(0)_{23} & ...\\
\rho(0)_{31} &\rho(0)_{32} & \rho(0)_{33} &  ...\\
... & ... & ... & ...
\end{bmatrix}
\rightarrow
\begin{bmatrix}
[\hat{\rho}(0)]_{m_{1}m_{1}} &[\hat{\rho}(0)]_{m_{1}m_{2}} & [\hat{\rho}(0)]_{m_{1}m_{3}} & ...\\
[\hat{\rho}(0)]_{m_{2}m_{1}} &[\hat{\rho}(0)]_{m_{2}m_{2}} & [\hat{\rho}(0)]_{m_{2}m_{3}} & ...\\
[\hat{\rho}(0)]_{m_{3}m_{1}} &[\hat{\rho}(0)]_{m_{3}m_{2}} & [\hat{\rho}(0)]_{m_{3}m_{3}} &  ...\\
... & ... & ... & ...
\end{bmatrix}.
\end{equation}
\noindent On the left hand side the density operator in the basis of the individual eigenfunctions of the non-relativisitic
Hamiltonian, $\{\Psi_{i}\}$. These non-relativistic wavefunctions can be grouped into spin multplets, each of which correspond
to a different spin. On the right hand side,  $[\hat{\rho}]_{m_{k}m_{k}}$ is block matrix representation of  $\hat{\rho}$ in 
the basis formed by the wavefunctions in the $k$th spin multiplet. \\ 


\noindent A Hamiltonian, $\hat{H}$, containing the magnetic field and all other desired perturbing operators,
and the associated time evolution operator, $\hat{U}(t)$, can be represented as
\begin{equation}
[\hat{H}] =
\begin{bmatrix}
[\hat{H}]_{m_{1}m_{1}} & [\hat{H}]_{m_{1}m_{2}}       & [\hat{H}]_{m_{1}m_{3}}  & ... \\
[\hat{H}]_{m_{2}m_{1}} & [\hat{H}]_{m_{2}m_{2}}       & [\hat{H}]_{m_{2}m_{3}}  & ... \\
[\hat{H}]_{m_{3}m_{1}} & [\hat{H}]_{m_{3}m_{2}}       & [\hat{H}]_{m_{3}m_{3}}  & ... \\
...      & ...     & ...     & ... \\
\end{bmatrix}
\rightarrow
[\hat{U}] =
\begin{bmatrix}
[\hat{U}(t)]_{m_{1}m_{1}} & [\hat{U}(t)]_{m_{1}m_{2}}       & [\hat{U}(t)]_{m_{1}m_{3}}  & ... \\
[\hat{U}(t)]_{m_{2}m_{1}} & [\hat{U}(t)]_{m_{2}m_{2}}       & [\hat{U}(t)]_{m_{2}m_{3}}  & ... \\
[\hat{U}(t)]_{m_{3}m_{1}} & [\hat{U}(t)]_{m_{3}m_{2}}       & [\hat{U}(t)]_{m_{3}m_{3}}  & ... \\
...      & ...     & ...     & ... \\
\end{bmatrix}
\end{equation}
\noindent The off-diagonal blocks govern the interaction between the different spin multiplets.\\

\noindent If the system is to be described as multiple seperate spins, then in order to
predict the spectra, it would be necessary to calculate the precession frequency of each of
these spins, and a method of weighting the contribution of this spin to the spectrum. One way
of accomplishing this would be to take the partial trace of


 we are interested in the expectation value, $\langle E_{1} \rangle$, associated with the 
then we may write:
\begin{equation}
\langle E_{1}(t) \rangle =
Tr\{ [\hat{H}]_{11}[\rho]_{11}(t) \} = 
Tr\{ [\hat{H}]_{11}] \sum_{ij} [\hat{U}(t)]_{1i}[\hat{\rho}(0)]_{ij} [\hat{U}^{\dagger}(t)]_{j1}\}.
\end{equation}
\noindent Which can be written as:
\begin{equation*}
\langle E_{1}(t) \rangle =
Tr\{ [\hat{H}]_{m_{1}m_{1}} [\hat{U}(t)]_{m_{1}m_{1}}[\rho(0)]_{m_{1}m_{1}} [\hat{U}^{\dagger}(t)]_{m_{1}m_{1}}\} 
\end{equation*}
\begin{equation*}
+ \sum_{k>1}Tr\{ [\hat{H}]_{m_{1}m_{1}} [\hat{U}(t)]_{m_{1}m_{k}}[\rho(0)]_{m_{k}m_{k}} [\hat{U}^{\dagger}(t)]_{m_{k}m_{1}}\}.
\end{equation*}
\begin{equation}
+ \sum_{\substack{k, l > 1 \\ k\neq l } }
Tr\{ [\hat{H}]_{m_{1}m_{1}} [\hat{U}(t)]_{m_{1}m_{k}}[\rho(0)]_{m_{k}m_{l}} [\hat{U}^{\dagger}(t)]_{m_{l}m_{1}}\}.
\end{equation}
\noindent The first term is just the standard expression for magnetic resonance tensors,
whilst the remaining terms may be interpreted as a corrections due to the excited multiplets being mixed into the ground state.
The second only includes density matrices corresponding to specific states, whilst the third term includes
transition density matrices, $[\rho]_{m_{i}m_{j}}$, the interpretation of which I am uncertain.\\

\noindent In most cases ensemble density operators, $\hat{\tilde{\rho}}$, will be necessary:
\begin{equation}
\hat{\tilde{\rho}} = \sum_{i}exp\Big[\frac{-\epsilon_{i}}{kT}\Big]\hat{\rho}_{i} 
\end{equation}
\noindent where $\epsilon_{i}$ is the energy eigenvalue of the $i$th eigenfunction of the \emph{relativistic} Hamiltonian. 


\end{document}
