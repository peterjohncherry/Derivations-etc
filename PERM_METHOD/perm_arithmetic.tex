\documentclass[12pt]{article}
\usepackage[utf8x]{inputenc}
\usepackage[english]{babel}
\usepackage[T1]{fontenc}
\usepackage{color}
\usepackage{wrapfig}
\usepackage{amsmath}
\usepackage{amssymb}
\usepackage{textcomp}
\usepackage{array}
\usepackage{booktabs}
\usepackage{subfigure}
\usepackage[font=small,format=plain,labelfont=bf,up,textfont=it,up]{caption}
\usepackage{longtable}
\usepackage{calc}
\usepackage{setspace}
\usepackage{multirow}
\usepackage{hhline}
\usepackage{ifthen}
\usepackage{lscape}
\usepackage{relsize}
\usepackage{bbold}
\usepackage{mathtext}
\usepackage{pdfpages}
\usepackage{geometry}
 \geometry{
 a4paper,
 total={170mm,257mm},
 left=35mm,
 top=25mm,
 bottom=25mm,
 right=20mm,
 }
\usepackage{cite}
\linespread{1.3}

\begin{document}

Have tensor

\begin{equation}
A_{ijkl...}
\end{equation}

\noindent Have a set of possible ranges each index may take. Define a set of numbers, $\{r\}_{r=0,..,N_{ranges}} $,
each one of which corresponds to a different range.\\

\noindent Write a block as
\begin{equation}
A_{ijkl...}^{r_{i} r_{j}r_{k}r_{l}...}
\end{equation}

\noindent Any given block can be uniquely identified with a number, $B$, in base $N_{ranges}$:

\begin{equation}
B = \sum_{q_{p}}^{N_ids} r_{q_{p}}N_{ranges}^{q_{p}}
\end{equation}

\noindent where $N_{ids}$ is the number of indexes. Here $q_{p}$ is the position of index $q$. E.g., 
for $A_{ijqkl}$  $q_{p} = 2$.  We now write a block as
\begin{equation}
A_{ijkl...}^{B}
\end{equation}

\noindent Permutation operations on the list of ranges can now be defined aritmetically. For example,
swapping the ranges of indexes $i$ and $j$ has the following effect on the Block number:
 
\begin{equation}
swap_{ij}(B)  = B  - r_{i_{p}}(N_{ranges}^{i_{p}} - N_{ranges}^{j_{p}}) + r_{j_{p}}(N_{ranges}^{j_{p}} - N_{ranges}^{i_{p}})
\end{equation}

\end{document}
